\documentclass[a4paper,12pt]{article}

\title{Ryder QFTゼミ資料}
\date{\today}
\author{Max Miyazaki}

\usepackage{amsmath}
\usepackage{amssymb}
\usepackage{ascmac}
\usepackage{amsfonts}
\usepackage{color}
\usepackage[dvipdfmx]{graphicx}
\usepackage{geometry}
\geometry{a4paper, margin=1in}
\usepackage{float}
\usepackage{bm}


% Define braket-like commands
\newcommand{\bra}[1]{\left\langle #1\right|}
\newcommand{\ket}[1]{\left|#1\right\rangle}
\newcommand{\braket}[2]{\left\langle #1\middle|#2\right\rangle}
\newcommand{\brakets}[3]{\left\langle #1\middle| #2 \middle|#3 \right\rangle}


\begin{document}
\maketitle
\section*{\textrm{5章 経路積分}}
\subsection*{\textrm{5.4 汎関数微分}}
プロパゲータ
\begin{equation*}
    \braket{x_f t_f}{x_i t_i} = \int \mathcal{D}x \exp\left[ \frac{i}{\hbar}\int_{t_i}^{t_f} L(x, \dot{x})dt \right]
\end{equation*}
のような量は関数積分であり, 積分はすべての関数 $x(t)$ に対して行われる. 左辺は数値なので積分は各関数 $x(t)$ に数値を関連付ける. この積分は汎関数と呼ばれ, すべての点のおける関数 $x(t)$ の値に依存することは明らかである. これを短く書くと以下のようになる.
\begin{equation*}
    \textrm{関数} : \textrm{数} \rightarrow \textrm{数}\tag{5.50}
\end{equation*}
関数 $ f(t) = t^2 + 2t $ のように, 各独立変数に対して値(つまり数値)を返すものを関数と呼ぶ. $ t $ に値を与えると, $ f $ の値を計算できる. 略記法として次のように書ける.

\begin{equation*}
\text{function: number} \rightarrow \text{number}. \tag{5.51}
\end{equation*}

数学的な記法では数値は実数空間 $ \mathbb{R} $ に属するので, 関数は次の写像を定義する.

\begin{equation*}
\text{function: } \mathbb{R} \rightarrow \mathbb{R}. \tag{5.52}
\end{equation*}

場合によっては関数がベクトル量であることもある. 例えば電場 $ \mathbf{E} $ は $ \mathbb{R}^3 $ に属し, 空間の3次元の各点に電場を割り当てる. したがって, 次の写像を定義する.

\begin{equation*}
\text{function: } \mathbb{R}^3 \rightarrow \mathbb{R}^3. 
\end{equation*}

一方で, スカラー関数 $ \phi(x) $ は次の写像を定義する.

\begin{equation*}
\text{function: } \mathbb{R}^3 \rightarrow \mathbb{R}.
\end{equation*}

一般的に次の定義を持つ.

\begin{equation*}
\text{function: } \mathbb{R}^n \rightarrow \mathbb{R}^m. \tag{5.53}
\end{equation*}

関数は連続であり, 正確には $ n $ 回微分可能である. 物理では通常, 無限回微分可能な関数に着目する. 基礎となる座標空間は多様体 $ M $ です(例えば $ \mathbb{R} $ や3次元ユークリッド空間の $ \mathbb{R}^3 $). 関数は $ C^n(M) $ と記され、無限回微分可能な関数の場合, $ C^\infty(M) $ と記される. したがって, 式(5.50)に従うと関数汎関数は次の写像を定義する.

\begin{equation*}
\text{functional: } C^\infty(M) \rightarrow \mathbb{R}. \tag{5.54}
\end{equation*}

ここで明確にしておくべきだが, 汎関数は関数の関数ではない. これはもちろん関数そのものである. 汎関数 $ F $ は、関数 $ f $ の関数として角括弧で記されることが一般的で, つまり $ F[f] $ と表記される.\\

次に汎関数微分を定義する. 通常の微分に類推して関数 $ f(y) $ に対する汎関数 $ F[f] $ の微分は次のように定義される.

\begin{equation*}
\frac{\delta F[f(x)]}{\delta f(y)} = \lim_{\epsilon \to 0} \frac{F[f(x) + \epsilon \delta(x - y)] - F[f(x)]}{\epsilon}. \tag{5.55}
\end{equation*}

具体的な例として次の汎関数を考えてみる.

\begin{equation*}
F[f] = \int f(x) \, dx. \tag{5.56}
\end{equation*}

そのとき, 次のようになる.

\begin{equation*}
\frac{\delta F[f]}{\delta f(y)} = \lim_{\epsilon \to 0} \frac{\left( \int \left( f(x) + \epsilon \delta(x - y) \right) dx \right) - \int f(x) dx}{\epsilon}
= \int \delta(x - y) dx = 1. \tag{5.57}
\end{equation*}

もう一つの例として次の汎関数を考える.

\begin{equation*}
F_x[f] = \int G(x, y) f(y) dy. \tag{5.58}
\end{equation*}

ここで, 左辺の $ x $ はパラメータとして扱われる. そのとき,

\begin{align*}
\frac{\delta F[f]}{\delta f(z)} 
&= \lim_{\epsilon \to 0} \frac{\int \left( G(x, y) f(y) + \epsilon G(x, y) \delta(y - z) \right) dy - \int G(x, y) f(y) dy}{\epsilon} \\
&= \int G(x, y) \delta(y - z) dy \\
&= G(x, z). \tag{5.59}
\end{align*}




\end{document}