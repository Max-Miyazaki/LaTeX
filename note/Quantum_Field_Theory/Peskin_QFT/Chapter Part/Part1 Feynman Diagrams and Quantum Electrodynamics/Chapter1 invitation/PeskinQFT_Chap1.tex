\documentclass[a4paper,12pt]{article}

\title{Part1 Feynman Diagrams and Quantum Electrodynamics\\
1. Invitation: Pair Production in $e^+ e^-$ Annihilation}
\date{各種SNS\\
    X (旧 Twitter): \href{https://x.com/miya_max_study}{@miya\_max\_study}\\
    Instagram : \href{https://www.instagram.com/daily_life_of_miya/}{@daily\_life\_of\_miya}\\
    YouTube : \href{https://www.youtube.com/@miya-max-active}{@miya-max-active}
    }
\author{Max Miyazaki}

\usepackage{amsmath}
\usepackage{amssymb}
\usepackage{ascmac}
\usepackage{amsfonts}
\usepackage{color}
\usepackage{graphicx}
\usepackage{float}
\usepackage{bm}
\usepackage[listings]{tcolorbox}
\usepackage{abstract}
\usepackage{tikz}
\usetikzlibrary{decorations.markings}
\usepackage{indentfirst}
\usepackage[utf8]{inputenc}
\usepackage{hyperref}
\usepackage{fix-cm}
\usepackage{wrapfig}
\hypersetup{%
 setpagesize=false,
 bookmarksnumbered=true,
 colorlinks=true,
 linkcolor=blue
}
\pagenumbering{arabic}
\usepackage{url}
% Define braket-like commands
\newcommand{\bra}[1]{\left\langle #1\right|}
\newcommand{\ket}[1]{\left|#1\right\rangle}
\newcommand{\braket}[2]{\left\langle #1\middle|#2\right\rangle}
\newcommand{\brakets}[3]{\left\langle #1\middle| #2 \middle|#3 \right\rangle}

%\newcommand{\tcb}[2]{\begin{tcolorbox}[title={\textcolor{white}{#1}}, opacitybacktitle = 0, colframe=white!40!black]#2
%    \end{tcolorbox}}



\renewcommand{\arraystretch}{2.1}

%\numberwithin{equation}{section}

\setlength{\textwidth}{16cm}
\setlength{\textheight}{24cm}
\setlength{\oddsidemargin}{0cm}
\setlength{\evensidemargin}{0cm}
\setlength{\topmargin}{-2cm}

\begin{document}
\maketitle

\vspace{1cm}
\begin{abstract}
    Peskin Quantum Field Theory の第1章をまとめたものです. メモ書き程度に色々追記がありますが、計算等については原書を参照してください.
\end{abstract}


\newpage
\section{\textrm{Pair Production in $e^+ e^-$ Annihilation}}
1部の目的は場の量子論の基本的な計算手法である Feynman diagram を定式化すること. Feynman diagram は電子と光子の量子論である量子電磁力学(QED)の計算においても非常に重要な役割を果たす.\par
量子電磁気学 (Quantum Electrodynamics, QED) は現時点では物理学の基礎理論として最も完成されたものと言える. QED は相対論的不変性の要請により本質的に決まり, Maxwell 方程式と Dirac 方程式の2つの方程式によって定式化される. これらの方程式の解は陽子の100倍程度小さいスケールにまで及ぶ微視的領域でも電磁気現象を予言する.\par
Feynman diagram はこの美しい理論と同様に等しい計算手法で, 電子と光子の相互作用を diagram を用いて量子力学的確率を数学的に表す.\par
本書の第一部では量子力学と相対論の基本的な原理に基づいて QED の理論と Feynman diagram の計算方法を議論し, 素粒子理論に現れる極めて重要な観測可能量を計算できるようになることを目指す. QED にの理解には2章から4章の形式的な議論を避けては通れず本書の目的を達成する前に何をやっているのか見失う可能性があるので, 本章では先に簡単な QED 過程を考えて多くの特徴を物理的な直観から理解することを試みる. その中で生じた疑問は5章で Feynman diagram を活用した同じ QED 過程の議論に戻ってくれば解決されるはずだ.

\end{document}
