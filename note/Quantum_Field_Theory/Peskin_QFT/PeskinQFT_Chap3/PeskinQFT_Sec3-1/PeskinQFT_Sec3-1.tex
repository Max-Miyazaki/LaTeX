\documentclass[a4paper,12pt]{article}

\title{Chapter 3. The Dirac Field\\
3-1. Lorentz Invariance in Wave Equations}
\date{各種SNS\\
    X (旧 Twitter): \href{https://x.com/miya_max_study}{@miya\_max\_study}\\
    Instagram : \href{https://www.instagram.com/daily_life_of_miya/}{@daily\_life\_of\_miya}\\
    YouTube : \href{https://www.youtube.com/@miya-max-active}{@miya-max-active}
    }
\author{Max Miyazaki}

\usepackage{amsmath}
\usepackage{amssymb}
\usepackage{ascmac}
\usepackage{amsthm}
\usepackage{amsfonts}
\usepackage{enumitem}
\usepackage{color}
\usepackage[dvipdfmx]{graphicx}
\usepackage{float}
\usepackage{bm}
\usepackage{here}

\usepackage{abstract}
\usepackage{tikz}
\usetikzlibrary{shapes.geometric, arrows.meta, positioning}
\usepackage{indentfirst}
\usepackage[utf8]{inputenc}
\usepackage{fix-cm}
\usepackage{wrapfig}
\pagenumbering{arabic}
\usepackage{url}
\usepackage{xcolor}
\usepackage[most]{tcolorbox}
\usepackage{framed}
\usepackage[dvipdfmx]{hyperref}
\hypersetup{
 setpagesize=false,
 bookmarksnumbered=true,
 colorlinks=true,
 linkcolor=blue
}

% Define braket-like commands
\newcommand{\bra}[1]{\left\langle #1\right|}
\newcommand{\ket}[1]{\left|#1\right\rangle}
\newcommand{\braket}[2]{\left\langle #1\middle|#2\right\rangle}
\newcommand{\brakets}[3]{\left\langle #1\middle| #2 \middle|#3 \right\rangle}

\renewcommand{\arraystretch}{2.1}


\setlength{\textwidth}{16cm}
\setlength{\textheight}{25cm}
\setlength{\oddsidemargin}{0cm}
\setlength{\evensidemargin}{0cm}
\setlength{\topmargin}{-2cm}

\begin{document}
\maketitle

\vspace{1cm}
\begin{abstract}
    このノートはPeskin\&Schroederの``An Introduction to Quantum Field Theory''の第3章の1節をまとめたものである. 要点や個人的な追記, 計算ノート的なまとめを行っているが, それらはすべて原書の内容を出発点としている. 参考程度に使っていただきたいが, このノートは私の勉強ノートであり, そのままの内容をそのまま鵜呑みにすると間違った理解を招く可能性があることをご了承ください. ぜひ原著を手に取り, その内容をご自身で確認していただくことを推奨します. てへぺろ v$({\hat{\cdot}_\partial \hat{\cdot}})$v
\end{abstract}
    
    

\newpage
\color{blue}
\section*{概要}
\begin{itemize}
  \item \textbf{Lorentz 不変性の定義}:\\
  微分方程式 $\mathcal{D}\phi = 0$ が Lorentz 不変とは, $\phi(x)$ がこの方程式を満たすならば,Lorentz変換後の場 $\phi'(\Lambda x)$ も同じ方程式を満たすこと.

  \item \textbf{ Lagrangian による不変性の保証}:\\
  方程式が Lorentz スカラーな Lagrangian から導かれるなら, Lorentz 不変性は自動的に満たされる.

  \item \textbf{スカラー場の Lorentz 変換}:\\
  スカラー場は以下のように変換される:
  \begin{equation*}
    \phi(x) \to \phi'(\Lambda x) = \phi(\Lambda^{-1}x)
  \end{equation*}

  \item \textbf{Klein-Gordon 場の例}:\\
  Lagrangian
  \begin{equation*}
    \mathcal{L} = \frac{1}{2} \partial_\mu \phi\, \partial^\mu \phi - \frac{1}{2} m^2 \phi^2
  \end{equation*}
  は Lorentz スカラーであり, 対応する方程式も不変.

  \item \textbf{多成分場の扱い}:\\
  ベクトル場 (例: $j^\mu$, $A^\mu$) は方向性を持つため, Lorentz 変換には行列表現が必要:
  \begin{equation*}
    V^\mu(x) \to \Lambda^\mu_{\ \nu} V^\nu(\Lambda^{-1}x)
  \end{equation*}

  \item \textbf{Maxwell 方程式の Lorentz 不変性}:\\
  方程式 $\partial^\nu F_{\mu\nu} = 0$ は, Lagrangian
  \begin{equation*}
    \mathcal{L}_\text{Maxwell} = -\frac{1}{4} F_{\mu\nu} F^{\mu\nu}
  \end{equation*}
  から導かれ, Lorentz 不変である.

  \item \textbf{より一般の Lorentz 変換表現}:\\
  多成分場 $\phi^a$ に対しては, Lorentz 変換行列 $M_{ab}(\Lambda)$ によって
  \begin{equation*}
    \Phi_a(x) \to M_{ab}(\Lambda)\, \Phi_b(\Lambda^{-1}x)
  \end{equation*}
  と変換される.

  \item \textbf{スピノル場の導入準備}:\\
  スピン1/2粒子の記述には, これらの一般化された表現 (特にディラック表現) を用いる必要がある.
\end{itemize}
\newpage
\color{black}
\section*{3.1 Lorentz Invariance in Wave Equations}
「ある方程式が相対論的に不変 (Lorentz 不変) である」とはどういう意味か?\\
$\phi$ が場 (あるいは場の集合体) であり, $\mathcal{D}$ が何らかの微分作用素であるとき, $\mathcal{D}\phi = 0$ が相対論的不変であるとは, 
\begin{itemize}
    \item『$\phi(x)$ がこの方程式を満たしているときに, 座標変換 (回転またはブースト) によって別の座標系に移ったとしても, 変換後の場 $\phi'(x)$ も同じ形の方程式を満たす』

\hspace{2cm}$\Longleftrightarrow$

    \item 物理的にすべての粒子・場を一様に回転またはブーストしても, もとの方程式 $\mathcal{D}\phi = 0$ は変わらず成り立つ.
\end{itemize}
例として, Klein-Gordon理論を考える. 任意の Lorentz 変換は
\begin{equation*}
x^\mu \to x'^\mu = \Lambda^\mu_{\ \nu} x^\nu \tag{3.1}
\end{equation*}
スカラー場の最も単純な変換法則は
\begin{equation*}
\phi(x) \to \phi'(x) = \phi(\Lambda^{-1} x) \label{3.2}\tag{3.2}
\end{equation*}
\color{blue}
変換前の場の配位を $\phi(x)$, 変換後の場の配位を $\phi'(x)$ とする. 変換後の場の配位の $x'$ における値は変換前の場の配位の $x$ における値なので,
\begin{equation*}
  \phi'(x') = \phi(x). \tag{3-1.a1}
\end{equation*}
つまり,
\begin{equation*}
  \phi'(\Lambda x) = \phi(x). \tag{3-1.a2}
\end{equation*}
よって, $\Lambda x$ を新たに $y$ とおくと,
\begin{equation*}
  \phi'(y) = \phi(\Lambda^{-1}y). \tag{3-1.a3}
\end{equation*}





\color{black}

この変換が Klein-Gordon ラグランジアンの形を保つことを確認する.
\begin{equation*}
  \mathcal{L} = \frac{1}{2}(\partial_\mu \phi(x))^2 + \frac{1}{2}m^2\phi(x)^2
\end{equation*}
式 \eqref{3.2} により, 質量項 $\frac{1}{2} m^2 \phi^2(x)$ は単に点 $\Lambda^{-1}x$ に移るだけである. $\partial_\mu \phi(x)$ の変換は
\begin{equation*}
\partial_\mu \phi(x) \rightarrow \partial{\textcolor{blue}{'}}_\mu \left( \phi(\Lambda^{-1}x) \right) = (\Lambda^{-1})^\nu_{\ \mu} (\partial_\nu \phi)(\Lambda^{-1}x). \label{3.3}\tag{3.3}
\end{equation*}

\color{blue}
\begin{proof}
\eqref{3.3} の導出:
\begin{equation*}
  x'^\rho = \Lambda^\rho_{\ \nu} x^\nu  \to  x^{\nu} = {(\Lambda^{-1})^{\nu}}_{\rho} x'^{\rho} \tag{3-1.b1}
\end{equation*}
ここで, $\partial'_{\mu}$ を合成関数の微分として考える:
\begin{align*}
  \frac{\partial}{\partial x'^{\mu}} \phi(\Lambda^{-1}x) &= \frac{\partial}{\partial x'^{\mu}} \phi(\Lambda^{-1}x) \tag{3-1.b2} \\
  &= \frac{\partial x^{\nu}}{\partial x'^{\mu}} \frac{\partial}{\partial x^{\nu}} \phi(\Lambda^{-1}x) \tag{3-1.b3} \\
  &= {(\Lambda^{-1})^{\nu}}_{\rho} \frac{\partial x'^{\rho}}{\partial x'^{\mu}} \frac{\partial}{\partial x^{\nu}} \phi(\Lambda^{-1}x) \tag{3-1.b4} \\
  &= {(\Lambda^{-1})^{\nu}}_{\rho} \delta^{\rho}_{\mu} \frac{\partial}{\partial x^{\nu}} \phi(\Lambda^{-1}x) \tag{3-1.b5} \\
  &= {(\Lambda^{-1})^{\nu}}_{\mu} \frac{\partial}{\partial x^{\nu}} \phi(\Lambda^{-1}x) \tag{3-1.b6} \\
  &= {(\Lambda^{-1})^{\nu}}_{\mu} \partial_\nu \phi(\Lambda^{-1}x) \tag{3-1.b7}
\end{align*}
\end{proof}
\color{black}
計量テンソル $g^{\mu\nu}$ は Lorentz 不変なので, $\Lambda^{-1}$ の行列は以下の恒等式を満たす:
\begin{equation*}
(\Lambda^{-1})^\rho_{\ \mu} (\Lambda^{-1})^\sigma_{\ \nu} g^{\mu\nu} = g^{\rho\sigma}. \label{3.4}\tag{3.4}
\end{equation*}
この関係を用いて, Klein-Gordon ラグランジアンの運動項の変換を計算すると:
\begin{align*}
(\partial_\mu \phi(x))^2 &\rightarrow g^{\mu\nu} (\partial{\textcolor{blue}{'}}_\mu \phi'(x)) (\partial{\textcolor{blue}{'}}_\nu \phi'(x)) \\
&= g^{\mu\nu} \left[ (\Lambda^{-1})^\rho_{\ \mu} \partial_\rho \phi \right] \left[ (\Lambda^{-1})^\sigma_{\ \nu} \partial_\sigma \phi \right](\Lambda^{-1}x) \\
&= g^{\rho\sigma} (\partial_\rho \phi)(\partial_\sigma \phi)(\Lambda^{-1}x) \\
&= (\partial_\mu \phi)^2(\Lambda^{-1}x).
\end{align*}
したがって, ラグランジアン全体はスカラーとして変換される:
\begin{equation*}
\mathcal{L}(x) \rightarrow \mathcal{L}(\Lambda^{-1}x). \tag{3.5}
\end{equation*}

ラグランジアン $\mathcal{L}$ を時空上で積分して得られる作用 $S$ は Lorentz 不変である.

同様の計算により, 運動方程式も不変であることが示される:
\begin{align*}
(\partial\textcolor{blue}{'}^2 + m^2)\phi'(x) &= \left[ (\Lambda^{-1})^\nu_{\ \mu} \partial_\nu (\Lambda^{-1})^\sigma_{\ \mu} \partial_\sigma + m^2 \right] \phi(\Lambda^{-1}x) \\
&= \left[ g^{\nu\sigma} \partial_\nu \partial_\sigma + m^2 \right] \phi(\Lambda^{-1}x) = 0.
\end{align*}
この変換法則 \eqref{3.2} は, 1成分の場に対して可能な最も単純なもので, 多成分場に対してはより複雑な変換法則が必要である. たとえば, 4元電流密度 $j^\mu(x)$ やベクトルポテンシャル $A^\mu(x)$ は, 単に空間的位置だけでなく「向き」も持っているので, 回転やブーストの際にその向きも変換されねばならない.

3次元回転下では:
\begin{equation*}
V^i(x) \to R^i_{\ j} V^j(R^{-1}x)
\end{equation*}
\color{blue}
回転行列 $R \in \text{SO}(3)$ は空間座標 $x^i$ に対して:
\begin{equation*}
  x'^i = R^i_{\ j} x^j \to  x^j = (R^{-1})^j_{\ i} x'^i \tag{3-1.c1}
\end{equation*}
ベクトル場の「幾何学的な意味」から出発すると, 座標系が回転して観測点が $x \to x' = R x$ となったとき, ベクトルそのものも回転により変換される:
\begin{equation*}
  V'^i(x') = R^i_{\ j}V^j(x) \tag{3-1.c2}
\end{equation*}
これを「$x'$ における新しい場 $V'^i(x')$ を, 元の場 $V^j(x)$ を使って記述する」形式にすると:
\begin{equation*}
  V^i(x) \to V'^i(x') = R^i_{\ j}V^j(R^{-1}x') \tag{3-1.c3}
\end{equation*}
\color{black}

Lorentz 変換下では:
\begin{equation*}
V^\mu(x) \to \Lambda^\mu_{\ \nu} V^\nu(\Lambda^{-1}x).
\end{equation*}
\color{blue}
※ローレンツ変換で4元ベクトル場 $V^\mu(x)$ もベクトルとして変換されるため.
\vskip\baselineskip
\color{black}
任意のランクのテンソル場もベクトルから構成でき, その変換法則にはより多くの $\Lambda$ の因子が現れる. 例えば Maxwell 方程式は次のように表せる:
\begin{equation*}
\partial^\nu F_{\mu\nu} = 0 \quad \text{または} \quad \Box A_\mu - \partial_\mu(\partial^\nu A_\nu) = 0. \tag{3.6}
\end{equation*}

これは Lagrangian
\begin{equation*}
\mathcal{L}_\text{Maxwell} = -\frac{1}{4} F_{\mu\nu} F^{\mu\nu} \tag{3.7}
\end{equation*}
から導かれ, Lorentz 不変である.

\color{blue}
\begin{proof}
$\partial^\nu F_{\mu\nu} = 0$ の Lorentz 不変性を示す.
\begin{align*}
  \partial'^\nu F'_{\mu\nu} &= (\Lambda^{-1})^\nu_{\ \rho} \partial^\rho (\Lambda^\sigma_{\ \mu} \Lambda^\lambda_{\ \nu} F_{\sigma\lambda}) \tag{3-1.d1} \\
  &= (\Lambda^{-1})^\nu_{\ \rho} \Lambda^\sigma_{\ \mu} \Lambda^\lambda_{\ \nu} \partial^\rho F_{\sigma\lambda} \tag{3-1.d2} \\
  &= \Lambda_\rho^{\ \nu} \Lambda^\sigma_{\ \mu} \Lambda^\lambda_{\ \nu} \partial^\rho F_{\sigma\lambda} \quad (\because (\Lambda^{-1})^\nu_{\ \rho} = \Lambda_\rho^{\ \nu}) \tag{3-1.d3} \\
  &= \Lambda^\sigma_{\ \mu} \delta^{\lambda}_{\rho} \partial^\rho F_{\sigma\lambda} \quad (\because \Lambda_\rho^{\ \nu} \Lambda^\lambda_{\ \nu} = \delta^{\lambda}_{\rho}) \tag{3-1.d4} \\
  &= \Lambda^\sigma_{\ \mu} (\partial^\lambda F_{\sigma\lambda}) = 0 \tag{3-1.d5} 
\end{align*}
\end{proof}
この計算で $(\Lambda^{-1})^\nu_{\ \rho} = \Lambda_\rho^{\ \nu}$ を用いたが, 非自明な式なので証明する.\\
\begin{proof}
\eqref{3.4} より,
\begin{align*}
  (\Lambda^{-1})^\rho_{\ \mu} (\Lambda^{-1})^\sigma_{\ \nu} g^{\mu\nu} &= g^{\rho\sigma} \tag{3-1.e1} \\
  g_{\lambda\rho}(\Lambda^{-1})^\rho_{\ \mu} (\Lambda^{-1})^\sigma_{\ \nu} g^{\mu\nu} &= g_{\lambda\rho}g^{\rho\sigma} \tag{3-1.e2}\\
  (\Lambda^{-1})_\lambda^{\ \nu} (\Lambda^{-1})^\sigma_{\ \nu} &= \delta^\sigma_{\lambda} \tag{3-1.e3}\\
  \Lambda_\lambda^{\ \nu} (\Lambda^{-1})_\lambda^{\ \nu} (\Lambda^{-1})^\sigma_{\ \nu} &= \Lambda_\lambda^{\ \nu} \delta^\sigma_{\lambda} \tag{3-1.e4}\\
  (\Lambda^{-1})^\sigma_{\ \nu} &= \Lambda_\sigma^{\ \nu} \tag{3-1.e5}
\end{align*}
\end{proof}

\color{black}

\begin{itemize}
  \item Lorentz 不変な方程式の特徴として, 各項の Lorentz 添字の構造 (縮約の仕方など) が同じであれば, その方程式は自然に Lorentz 不変になる.
  \item より一般的な Lorentz 不変な式を得るには, 場の変換則を体系的に調べるのが有効.
  \item 具体的な変換則として線形な場合を考える. つまり, $\Phi_a$ が $n$ 成分の多重項であるとすると Lorentz 変換 $\Lambda$ は, $n \times n$ 行列 $M(\Lambda)$ を用いて次のように変換される.
  \begin{equation*}
    \Phi_a(x) \to \Phi_a'(x) = M_{ab}(\Lambda) \Phi_b(\Lambda^{-1}x) \label{3.8}\tag{3.8}
  \end{equation*}
\end{itemize}
つまり, Lorentz 不変な理論を構築するには, 場の変換法則を定めて, それに従って不変な Lagrangian や方程式を構成すればよい.
\vskip\baselineskip
一般的な非線形変換も, これらの線形変換を元に構成できることが知られており, \eqref{3.8} から一般の変換を考える必要はない.\\
\color{blue}
※ 非慣性系の変換について, Lorentz 変換は慣性系から慣性系への変換なので非慣性系については適応できないが, 微小時間に分けて連続的に Lorentz 変換を適応すれば非慣性系の変換が再現できる.
\vskip\baselineskip
\color{black}
よって, 以降の議論では場の引数の変化を省略し, 変換を以下のように表す:
\begin{equation*}
  \Phi \to M(\Lambda)\Phi \label{3.9}\tag{3.9}
\end{equation*}
ではこの行列 $M(\Lambda)$ のとりうる形は何か?\\
基本的な制限は, 変換が群をなすこと. なので連続した2つの変換 $\Lambda$ と $\Lambda'$ に対し, 
\begin{equation*}
\Phi \rightarrow M(\Lambda')M(\Lambda)\Phi = M(\Lambda'')\Phi\quad (\Lambda'' = \Lambda'\Lambda) \tag{3.10}
\end{equation*}
が成り立つ必要がある. これは, 連続した2つの変換は1つの変換にまとめられることを意味する. このことから, $M(\Lambda)$ は Lorentz 群の表現(表現行列)でなければならない.\\
つまり, 行列 $M$ は Lorentz 群の $n$ 次元表現でなければならず, 言い換えると,
\begin{center}
  \textbf{『Lorentz 群の有限次元の行列表現はどのようなものか?』}という問いに帰着する.
\end{center}
\color{blue}
※無限次元でない理由:注目したいのはスカラー場・Dirac 場・ベクトル場などの「場」が物理的対象であり, それらは有限個の成分を持つため. これらの場が Lorentz 変換でどう変わるかは有限次元の表現で記述される. また, 有限次元表現は具体的な行列で書けるため, Lagrangian や作用を明示的に構成でき, 理論構築が容易になる. 無限次元を考える場合もあるが, それは状態空間そのものを議論したり高度な理論 (散乱理論や共形場理論など) を構築する場合に必要で, 基本的な場の変換法則の理解と構築には必要ない.
\vskip\baselineskip
\color{black}
\textbf{理解のためのステップ:}
\begin{itemize}
  \item \textbf{簡単な例:3次元回転群} \\
  回転群はスピンに応じて $j$ 次元の表現を持ち, スピン $s$ に対して $j = 2s + 1$ である.
  \vskip\baselineskip
  \color{red}
  そうなることを示す.
  \vskip\baselineskip
  \color{black}
  \item \textbf{スピン1/2の表現}:\\
  Pauli 行列 $\sigma^i$ を用いて, 次のように回転行列を構成:
  \begin{equation*}
  U = e^{-i\theta^i \sigma^i/2} \label{3.11}\tag{3.11}
  \end{equation*}
  $U$ は行列式1の $2\times 2$ 行列. $\theta^i$ は任意の3つのパラメータ, $\sigma^i$ は Pauli 行列である.
  \item \textbf{Lie代数と生成子} ※Lie代数の詳しい説明は省く:\\
  回転群において, 無限小回転に対応する生成子 $J^i$ は以下の交換関係を満たす:
  \begin{equation*}
  [J^i, J^j] = i\epsilon^{ijk}J^k \label{3.12}\tag{3.12}
  \end{equation*}
  \item \textbf{この生成子使って有限回転操作を指数関数で構成}:
  \begin{equation*}
  R = \exp(-i\theta^i J^i) \tag{3.13}
  \end{equation*}
  これが $\hat{\theta}$ 軸まわりに $|\theta|$ だけ回転させる演算子になる. 演算子 $J^i$ の交換関係が, これらの回転の乗法則を決める.
  \item \textbf{回転群の表現構築}:\\
  回転群の交換関係 (\eqref{3.12}) を満たす演算子の集合から, 指数関数により群の表現を構成できる. 例えば, 角運動量演算子の表現として,
  \begin{equation*}
  J^i \to \frac{\sigma^i}{2} \tag{3.14}
  \end{equation*}
  を用いると, \eqref{3.11} の回転群の表現が得られる. \textcolor{red}{このことを示す}
  \item \textbf{一般の連続群に対する原理:}\\
  任意の連続群に対して,
  \begin{itemize}
    \item 適切な交換関係を満たす群の生成子の行列表現を見つける.
    \item それを無限小変換の指数関数として構成すれば, 群の表現を構成できる.
  \end{itemize} 
  \item \textbf{Lorentz 群に対する適用:}\\
  Lorentz 群の生成子の交換関係を知りたい.\par
  $\Longrightarrow$回転群と同様に生成子を微分演算子として表現する.
  \item 3次元空間では, 角運動量演算子は次のように書ける:
  \begin{equation*}
    \mathbf{J} = \mathbf{x} \times (-i\nabla) \tag{3.15}
  \end{equation*}
  \item より一般には, 反対称テンソルとして次のように書ける:
  \begin{equation*}
    J^{ij} = -i(x^i \nabla^j - x^j \nabla^i)
  \end{equation*}
  4次元 Lorentz 変換への自然な拡張をすると,
  \begin{equation*}
    J^{\mu\nu} = i(x^\mu \partial^\nu - x^\nu \partial^\mu). \tag{3.16}
  \end{equation*}
  \item これら $J^{\mu\nu}$ は, Lorentz 群の3つの回転と3つのブーストを生成する.
  \item 微分演算子の交換子を計算すると, Lorentz 代数の交換関係が得られる:
  \begin{equation*}
    [J^{\mu\nu}, J^{\rho\sigma}] = i \left( g^{\nu\rho} J^{\mu\sigma}
      - g^{\mu\rho} J^{\nu\sigma}
      - g^{\nu\sigma} J^{\mu\rho}
      + g^{\mu\sigma} J^{\nu\rho} \right) \label{3.17}\tag{3.17}
  \end{equation*}
  \item この交換関係を満たす行列が, Lorentz 群の表現となる. そしてこの代数を表現する全ての行列は, この交換関係を満たさなければならない. この理解が正しいかどうか確認するために特定の表現で確認する.
  \item $4 \times 4$ 行列表現の例:
  \begin{equation*}
  (J^{\mu\nu})_{\alpha\beta} = i(\delta^\mu_\alpha g^\nu_\beta - \delta^\nu_\alpha g^\mu_\beta) \label{3.18}\tag{3.18}
  \end{equation*}
  $\mu$, $\nu$ はどの6つの行列を取るかを決める添字で, $\alpha$, $\beta$ はそれらの行列の成分を指定する添字である. \textcolor{red}{なぜ6つなのか記載する.}\\
  \eqref{3.18} が \eqref{3.17} を満たすことは簡単に確認できる.
  \color{blue}
  \vskip\baselineskip
  \begin{proof}
  \begin{align*}
    [J^{\mu\nu}, J^{\rho\sigma}] &= i \left( g^{\nu\rho} J^{\mu\sigma} - g^{\mu\rho} J^{\nu\sigma} - g^{\nu\sigma} J^{\mu\rho} + g^{\mu\sigma} J^{\nu\rho} \right) \\
  \end{align*}
  \end{proof}
  \vskip\baselineskip
  \color{black}
\end{itemize}

\vspace{1em}
\textbf{無限小 Lorentz 変換の表現}:

4元ベクトル $V^\alpha$ に対して、無限小変換を次のようにパラメータ化する:
\begin{equation*}
V^\alpha \rightarrow \left( \delta^\alpha_\beta - \frac{i}{2} \omega_{\mu\nu}(J^{\mu\nu})^\alpha_{\ \beta} \right)V^\beta \label{3.19}\tag{3.19}
\end{equation*}
$\omega_{\mu\nu}$ は反対称テンソルで, その成分は無限小変換のパラメータであり, 無限小角度を与える.
\begin{itemize}
  \item 例1:$\omega_{12} = -\omega_{21} = \theta$ の場合 \eqref{3.19}は,
  \begin{equation*}
  V \rightarrow 
  \begin{pmatrix}
  1 & 0 & 0 & 0 \\
  0 & 1 & -\theta & 0 \\
  0 & \theta & 1 & 0 \\
  0 & 0 & 0 & 1
  \end{pmatrix}
  V \tag{3.20}
  \end{equation*}
  これは $xy$ 平面内での無限小回転を表す.

  \item 例2:$\omega_{01} = -\omega_{10} = \beta$ とすると,
  \begin{equation*}
  V \rightarrow 
  \begin{pmatrix}
  1 & \beta & 0 & 0 \\
  \beta & 1 & 0 & 0 \\
  0 & 0 & 1 & 0 \\
  0 & 0 & 0 & 1
  \end{pmatrix}
  V \tag{3.21}
  \end{equation*}
  これは $x$ 方向の無限小ブーストを表す. その他の $\omega_{\mu\nu}$ の組み合わせに対しても, ブーストや回転を同様の方法で生成できる.
\end{itemize}

これらの無限小変換は、全ての Lorentz 変換(回転とブースト)を生成するための基本である。







\end{document}
