\documentclass[a4paper,12pt]{article}

\title{Chapter 3. The Dirac Field\\
3-4. Dirac Matrices and Dirac Fields Bilinears}
\date{各種SNS\\
    X (旧 Twitter): \href{https://x.com/miya_max_study}{@miya\_max\_study}\\
    Instagram : \href{https://www.instagram.com/daily_life_of_miya/}{@daily\_life\_of\_miya}\\
    YouTube : \href{https://www.youtube.com/@miya-max-active}{@miya-max-active}
    }
\author{Max Miyazaki}

\usepackage{amsmath}
\usepackage{amssymb}
\usepackage{ascmac}
\usepackage{amsthm}
\usepackage{amsfonts}
\usepackage{enumitem}
\usepackage{color}
\usepackage[dvipdfmx]{graphicx}
\usepackage{float}
\usepackage{bm}
\usepackage{here}

\usepackage{abstract}
\usepackage{tikz}
\usetikzlibrary{shapes.geometric, arrows.meta, positioning}
\usepackage{indentfirst}
\usepackage[utf8]{inputenc}
\usepackage{fix-cm}
\usepackage{wrapfig}
\pagenumbering{arabic}
\usepackage{url}
\usepackage{xcolor}
\usepackage[most]{tcolorbox}
\usepackage{framed}
\usepackage[dvipdfmx]{hyperref}
\hypersetup{
 setpagesize=false,
 bookmarksnumbered=true,
 colorlinks=true,
 linkcolor=blue
}

% Define braket-like commands
\newcommand{\bra}[1]{\left\langle #1\right|}
\newcommand{\ket}[1]{\left|#1\right\rangle}
\newcommand{\braket}[2]{\left\langle #1\middle|#2\right\rangle}
\newcommand{\brakets}[3]{\left\langle #1\middle| #2 \middle|#3 \right\rangle}

\renewcommand{\arraystretch}{2.1}


\setlength{\textwidth}{16cm}
\setlength{\textheight}{25cm}
\setlength{\oddsidemargin}{0cm}
\setlength{\evensidemargin}{0cm}
\setlength{\topmargin}{-2cm}

\begin{document}
\maketitle

\vspace{1cm}
\begin{abstract}
    このノートはPeskin\&Schroederの``An Introduction to Quantum Field Theory''の第3章の4節をまとめたものである. 要点や個人的な追記, 計算ノート的なまとめを行っているが, それらはすべて原書の内容を出発点としている. 参考程度に使っていただきたいが, このノートは私の勉強ノートであり, そのままの内容をそのまま鵜呑みにすると間違った理解を招く可能性があることをご了承ください. ぜひ原著を手に取り, その内容をご自身で確認していただくことを推奨します. てへぺろ v$({\hat{\cdot}_\partial \hat{\cdot}})$v
\end{abstract}
    
    

\newpage
\color{blue}
\section*{概要}
\begin{itemize}
  \item Dirac 場の双線形形式 $\bar{\psi} \Gamma \psi$ が Lorentz 変換の下でどのような性質を持つかを系統的に分類する. 双線形形式は$\psi$と$\bar{\psi}$を使って Lorentz 共変に構成される基本的な物理量のことで, Lorentz 不変な作用を構築するのに不可欠なものである.

  \item 任意の $4 \times 4$ 行列 $\Gamma$ は, 以下の16個の線形独立な行列の基底に展開できる:
  \begin{itemize}
    \item スカラー:$I$(単位行列) $\,\Rightarrow$ $\bar{\psi} \psi$
    \item ベクトル:$\gamma^\mu$ $\,\Rightarrow$ $\bar{\psi} \gamma^\mu \psi$
    \item テンソル:$\sigma^{\mu\nu} = \frac{i}{2} [\gamma^\mu, \gamma^\nu]$ $\,\Rightarrow$ $\bar{\psi} \sigma^{\mu\nu} \psi$
    \item 偽ベクトル:$\gamma^\mu \gamma^5$ $\,\Rightarrow$ $\bar{\psi} \gamma^\mu \gamma^5 \psi$
    \item 偽スカラー:$\gamma^5$ $\,\Rightarrow$ $\bar{\psi} \gamma^5 \psi$
  \end{itemize}

  \item これらの16成分は, スカラー、ベクトル、2階テンソル、擬ベクトル、擬スカラーとして Lorentz 変換する.
  各成分は物理的意味を持ち, 相互作用項や保存カレントの構成に重要である.

  \item $\gamma^5$ は以下のように定義される:
  \begin{align*}
    \gamma^5 &= i \gamma^0 \gamma^1 \gamma^2 \gamma^3
  \end{align*}
  この行列は以下の性質を持つ:
  \begin{align*}
    (\gamma^5)^\dagger = \gamma^5,\quad (\gamma^5)^2 = 1,\quad \{ \gamma^5, \gamma^\mu \} = 0
  \end{align*}

  \item $\gamma^5$ によって Dirac スピノールは左手・右手成分に分離できる:
  \begin{align*}
    \psi_L = \frac{1 - \gamma^5}{2} \psi,\quad \psi_R = \frac{1 + \gamma^5}{2} \psi
  \end{align*}
  これは Chiral 対称性の理解に不可欠である.

  \item Noether の定理により, 以下の保存カレントが得られる:
  \begin{align*}
    j^\mu &= \bar{\psi} \gamma^\mu \psi \quad \text{(電流)} \\
    j_5^\mu &= \bar{\psi} \gamma^\mu \gamma^5 \psi \quad \text{(軸電流)}
  \end{align*}
  \begin{itemize}
    \item $m \neq 0$ の場合:$j^\mu$ のみ保存.
    \item $m = 0$ の場合:$j^\mu$, $j_5^\mu$ ともに保存.
    \item $j^\mu_L = \bar{\psi}_L \gamma^\mu \psi_L$, $j^\mu_R = \bar{\psi}_R \gamma^\mu \psi_R$ も保存される.
  \end{itemize}

  \item 双線形形式の積には Fierz 恒等式が成立し, スピン構造の書き換えに使える.
  \begin{equation*}
    (\sigma^\mu)_{\alpha\beta} (\sigma_\mu)_{\gamma\delta} = 2 \epsilon_{\alpha\gamma} \epsilon_{\beta\delta}
  \end{equation*}
\end{itemize}

\newpage
\color{black}
\section*{3.4 Dirac Matrices and Dirac Fields Bilinears}
\begin{itemize}
  \item セクション3.2で$\bar{\psi} \psi$ が Lorentz スカラーであることを確認した.\\
  \color{blue}
  $\psi^* \psi$ では Lorentz 不変にならなかったので Dirac 共役 $\bar{\psi} = \psi^{\dagger}\gamma^0$ を定義していた.
  \color{black}
  \item また, $\bar{\psi} \gamma^\mu \psi$ は4-ベクトルである (Dirac Lagrangian の構築に使用).
  \item より一般的に, 任意の $4 \times 4$ 定数行列 $\Gamma$ に対して $\bar{\psi} \Gamma \psi$ の Lorentz 変換性を考える.
  \item $\Gamma$ は次の16個の反対称な $\gamma$ 行列の組み合わせを基底とする空間に展開できる:
  \begin{itemize}
    \item スカラー:$1$ \hfill (1個)
    \item ベクトル:$\gamma^\mu$ \hfill (4個)
    \item テンソル:$\gamma^{\mu\nu} = \frac{1}{2}[\gamma^\mu, \gamma^\nu] = -i \sigma^{\mu\nu}$ \hfill (6個)
    \item 3階テンソル:$\gamma^{\mu\nu\rho} = \gamma^{[\mu} \gamma^\nu \gamma^{\rho]}$ \hfill (4個)
    \item 4階テンソル:$\gamma^{\mu\nu\rho\sigma} = \gamma^{[\mu} \gamma^\nu \gamma^\rho \gamma^{\sigma]}$ \hfill (1個)
  \end{itemize}
  \item 合計で16個の独立な双線形形式が存在する.

  \item これらの行列の Lorentz 変換特性は計算しやすく, 例えば次のように変換される:
  \begin{align*}
    \bar{\psi} \gamma^{\mu\nu} \psi &\rightarrow \left( \bar{\psi} \Lambda^{-1}_{\frac{1}{2}} \right) \left( \tfrac{1}{2}[\gamma^\mu, \gamma^\nu] \right) \left( \Lambda_{\frac{1}{2}} \psi \right) \\
    &= \tfrac{1}{2} \bar{\psi} \left( \Lambda^{-1}_{\frac{1}{2}} \gamma^\mu \Lambda_{\frac{1}{2}} \Lambda^{-1}_{\frac{1}{2}} \gamma^\nu \Lambda_{\frac{1}{2}} - \Lambda^{-1}_{\frac{1}{2}} \gamma^\nu \Lambda_{\frac{1}{2}} \Lambda^{-1}_{\frac{1}{2}} \gamma^\mu \Lambda_{\frac{1}{2}} \right) \psi \\
    &= \Lambda^\mu_{\ \alpha} \Lambda^\nu_{\ \beta} \bar{\psi} \gamma^{\alpha\beta} \psi
  \end{align*}
  \color{blue}
  式 (3.29) より $\Lambda^{-1}_{\frac{1}{2}} \gamma^\mu \Lambda_{\frac{1}{2}} = \Lambda^\mu_{\ \alpha}\gamma^{\alpha}$ を用いて計算している.
  \color{black}
  \item 各双線形形式は, Lorentz 群の下で各ランクに応じた反対称テンソルとして変換する.
  \item 高階の $\gamma$ 行列 ($\gamma^{\mu\nu\rho}$, $\gamma^{\mu\nu\rho\sigma}$) は, 補助的な $\gamma^5$ を導入することで簡略化できる:
  \begin{equation*}
    \gamma^5 \equiv i\gamma^0 \gamma^1 \gamma^2 \gamma^3 = -\frac{i}{4!} \epsilon^{\mu\nu\rho\sigma} \gamma_\mu \gamma_\nu \gamma_\rho \gamma_\sigma \tag{3.68}
  \end{equation*}

  \item このとき, 次のような関係が成り立つ:
  \begin{equation*}
    \gamma^{\mu\nu\rho\sigma} = -i \epsilon^{\mu\nu\rho\sigma} \gamma^5, \quad
    \gamma^{\mu\nu\rho} = +i \epsilon^{\mu\nu\rho\sigma} \gamma_\sigma \gamma^5
  \end{equation*}

  \item $\gamma^5$ の主な性質は以下の通り:
  \begin{align*}
    (\gamma^5)^\dagger &= \gamma^5 \tag{3.69}\\
    (\gamma^5)^2 &= 1 \tag{3.70}\\
    \{ \gamma^5, \gamma^\mu \} &= 0 \tag{3.71}
  \end{align*}

  \item 特に $\{ \gamma^5, \gamma^\mu \} = 0$ より, $[\gamma^5, S^{\mu\nu}] = 0$ が導かれ,
        Dirac 表現が可約であることが示される (Schur's 補題により, 異なる固有値を持つ状態が混ざらない).

  \item 我々の基底では, $\gamma^5$ は以下のブロック対角形で表される:
  \begin{equation*}
    \gamma^5 =
    \begin{pmatrix}
    -1 & 0 \\
    0 & 1
  \end{pmatrix} \tag{3.72}
  \end{equation*}

  \item したがって, $\gamma^5$ の固有値 $-1$($+1$)を持つスピノールは,
        左手(右手)成分に対応し, 混ざることなく独立に変換される.

  \item $4 \times 4$ 行列の標準的な分類は次の通り:
  \begin{itemize}
    \item $1$:scalar(スカラー)・・・1個
    \item $\gamma^\mu$:vector(ベクトル)・・・4個
    \item $\sigma^{\mu\nu} = \frac{1}{2}[\gamma^\mu, \gamma^\nu]$:tensor(テンソル)・・・6個
    \item $\gamma^\mu \gamma^5$:pseudo-vector(擬ベクトル)・・・4個
    \item $\gamma^5$:pseudo-scalar(擬スカラー)・・・1個
    \item 合計:16個.
  \end{itemize}

  \item 擬ベクトル・擬スカラーとは:
  \begin{itemize}
    \item 連続的な Lorentz 変換の下では, それぞれベクトル・スカラーとして変換する.
    \item ただし, パリティ変換の下で符号が反転する (セクション3.6で詳述予定).
  \end{itemize}

  \item ベクトル・擬ベクトル双線形形式から以下の2つのカレントを構成できる:
  \begin{align*}
    j^\mu(x) &= \bar{\psi}(x) \gamma^\mu \psi(x) \\
    j^{\mu5}(x) &= \bar{\psi}(x) \gamma^\mu \gamma^5 \psi(x) \tag{3.73}
  \end{align*} 
  \item Dirac カレント $j^\mu = \bar{\psi} \gamma^\mu \psi$ の保存は, Dirac 方程式を用いて確認できる:
  \begin{align*}
    \partial_\mu j^\mu &= (\partial_\mu \bar{\psi}) \gamma^\mu \psi + \bar{\psi} \gamma^\mu \partial_\mu \psi \\
    &= (im\bar{\psi}) \psi + \bar{\psi}(-im\psi) = 0 \tag{3.74}
  \end{align*}
  よって $\psi(x)$ が Dirac 方程式を満たす限り, $j^\mu$ は常に保存される.

  \item $\psi$ を電磁場と結合した場合, この $j^\mu$ は電流密度として働く.

  \item 同様に $j^{\mu5} = \bar{\psi} \gamma^\mu \gamma^5 \psi$ に対して,
  \begin{equation*}
    \partial_\mu j^{\mu5} = 2im \bar{\psi} \gamma^5 \psi \tag{3.75}
  \end{equation*}
  $m = 0$ のとき, このカレント (軸ベクトルカレントと呼ばれる) も保存される.

  \item このとき, $j^\mu_L$, $j^\mu_R$ を以下のように定義すると便利である:
  \begin{equation*}
    j^\mu_L = \bar{\psi} \gamma^\mu \left( \frac{1 - \gamma^5}{2} \right) \psi, \quad
    j^\mu_R = \bar{\psi} \gamma^\mu \left( \frac{1 + \gamma^5}{2} \right) \psi \tag{3.76}
  \end{equation*}

  \item $m = 0$ のとき, これらは左手・右手粒子の電流密度となり, 互いに独立に保存される.

  \item $j^\mu(x)$ および $j^{\mu5}(x)$ は, 以下の2つの変換に対応する Noether カレントである:
  \begin{equation*}
    \psi(x) \rightarrow e^{i\alpha} \psi(x), \qquad
    \psi(x) \rightarrow e^{i\alpha \gamma^5} \psi(x)
  \end{equation*}

  \item 前者は質量項を含む Dirac Lagrangian の対称性. 後者はカイラル変換であり, Dirac Lagrangian の微分項のみに対称性を持ち, 質量項には対称性を持たない. よって, Noether の定理より, 軸ベクトルカレントの保存は $m = 0$ のときに限られる.

  \item Dirac 場の双線形形式の積には交換関係が成り立ち, Fierz 恒等式と呼ばれる.

  \item 本書では最も簡単な恒等式のみを使用し, 2成分 Weyl スピノールにより記述するのが簡単である (式(3.36) 参照).

  \item 基本的な恒等式 ($2 \times 2$ 行列 $\sigma^\mu$ に対する) は以下(式(3.41) で定義):
  \begin{equation*}
    (\sigma^\mu)_{\alpha\beta} (\sigma_\mu)_{\gamma\delta} = 2 \epsilon_{\alpha\gamma} \epsilon_{\beta\delta} \label{3.77}\tag{3.77}
  \end{equation*}
  ここで $\alpha, \beta, \gamma, \delta$ はスピノル添字, $\epsilon$ は反対称記号である. この関係は, $\alpha, \gamma$ が $\psi_L$ の Lorentz 表現に従って変換し, $\beta, \delta$ が $\psi_R$ の別の表現に従って変換することにより, 全体として Lorentz 不変量となることから理解できる. あるいは, \eqref{3.77} の全16成分を明示的に検証することもできる. この恒等式を, Dirac スピノール $u_1, u_2, u_3, u_4$ の右手成分 (つまり下側) に挿入することで, 以下の恒等式が得られる:
  \begin{align*}
    (\bar{u}_{1R} \sigma^\mu u_{2R})(\bar{u}_{3R} \sigma_\mu u_{4R}) &=
    2\epsilon_{\alpha\gamma} \bar{u}_{1R\alpha} \bar{u}_{3R\gamma} \epsilon_{\beta\delta} u_{2R\beta} u_{4R\delta}\\
    &= -(\bar{u}_{1R} \sigma^\mu u_{4R})(\bar{u}_{3R} \sigma_\mu u_{2R}) \label{3.78}\tag{3.78}
  \end{align*}
  \item \eqref{3.78} の非自明な関係は, 双線形形式の積がラベル $2 \leftrightarrow 4$ の交換に対して反対称であり, $1 \leftrightarrow 3$ の交換に対しても反対称であることを示す.

  \item 恒等式 \eqref{3.77} は $\bar{\sigma}^\mu$ に対しても成り立つため, 次のような関係が得られる:
  \begin{equation*}
    (\bar{u}_{1L} \bar{\sigma}^\mu u_{2L})(\bar{u}_{3L} \bar{\sigma}_\mu u_{4L}) = -(\bar{u}_{1L} \bar{\sigma}^\mu u_{4L})(\bar{u}_{3L} \bar{\sigma}_\mu u_{2L}) \tag{3.79}
  \end{equation*}

  \item Fierz 恒等式 \eqref{3.78} を以下の $\sigma^\mu$ と $\bar{\sigma}^\mu$ を結ぶ恒等式と組み合わせると便利:
  \begin{equation*}
    \epsilon_{\alpha\beta} (\sigma^\mu)_{\beta\gamma} = (\bar{\sigma}^{\mu T})_{\alpha\beta} \epsilon_{\beta\gamma} \tag{3.80}
  \end{equation*}

  \item 次の恒等式も用いる:
  \begin{equation*}
    \bar{\sigma}^\mu \sigma_\mu = 4 \tag{3.81}
  \end{equation*}

  \item これらを用いて, 次のような複雑な双線形形式の積を簡単に整理できる:
  \begin{align*}
  (\bar{u}_{1L} \bar{\sigma}^\mu \sigma^\nu \bar{\sigma}^\lambda u_{2L})(\bar{u}_{3L} \bar{\sigma}_\mu \sigma_\nu \bar{\sigma}_\lambda u_{4L})&= 2 \epsilon_{\alpha\gamma} \bar{u}_{1L\alpha} \bar{u}_{3L\gamma} \epsilon_{\beta\delta} (\sigma^\nu \bar{\sigma}^\lambda u_{2L})_\beta (\sigma_\nu \bar{\sigma}_\lambda u_{4L})_\delta \\
  &= 2 \epsilon_{\alpha\gamma} \bar{u}_{1L\alpha} \bar{u}_{3L\gamma} \epsilon_{\beta\delta}u_{2L\beta} (\sigma^\nu \bar{\sigma}^\lambda \sigma_\nu \bar{\sigma}_\lambda u_{4L})_\delta \\
  &= 2 \cdot (4)^2 \cdot \epsilon_{\alpha\gamma} \bar{u}_{1L\alpha} \bar{u}_{3L\gamma} \epsilon_{\beta\delta} u_{2L\beta} u_{4L\delta} \\
  &= 16 (\bar{u}_{1L} \bar{\sigma}^\mu u_{2L})(\bar{u}_{3L} \bar{\sigma}_\mu u_{4L}) \tag{3.82}
  \end{align*}

  \item Fierz 恒等式は, 4成分 Dirac スピノールと $4 \times 4$ Dirac 行列に対しても存在する.

  \item これらの導出にはより体系的なアプローチが必要であり, 後続問題 (Problem 3.6) において一般的手法とその応用例が提示される.
\end{itemize}










\end{document}
