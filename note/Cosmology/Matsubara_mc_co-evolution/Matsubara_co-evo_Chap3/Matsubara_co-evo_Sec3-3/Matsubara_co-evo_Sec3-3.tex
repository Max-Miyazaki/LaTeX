\documentclass[a4paper,12pt]{article}

\title{第3章 膨張宇宙の力学\\
3.3. 宇宙論パラメータ}
\date{各種SNS\\
    X (旧 Twitter): \href{https://x.com/miya_max_study}{@miya\_max\_study}\\
    Instagram : \href{https://www.instagram.com/daily_life_of_miya/}{@daily\_life\_of\_miya}\\
    YouTube : \href{https://www.youtube.com/@miya-max-active}{@miya-max-active}
    }
\author{Max Miyazaki}

\usepackage{amsmath}
\usepackage{amssymb}
\usepackage{ascmac}
\usepackage{amsthm}
\usepackage{amsfonts}
\usepackage{enumitem}
\usepackage{color}
\usepackage[dvipdfmx]{graphicx}
\usepackage{float}
\usepackage{bm}
\usepackage{here}

\usepackage{abstract}
\usepackage{tikz}
\usetikzlibrary{shapes.geometric, arrows.meta, positioning}
\usepackage{indentfirst}
\usepackage[utf8]{inputenc}
\usepackage{fix-cm}
\usepackage{wrapfig}
\pagenumbering{arabic}
\usepackage{url}
\usepackage{xcolor}
\usepackage[most]{tcolorbox}
\usepackage{framed}
\usepackage[dvipdfmx]{hyperref}
\hypersetup{
 setpagesize=false,
 bookmarksnumbered=true,
 colorlinks=true,
 linkcolor=blue
}

% Define braket-like commands
\newcommand{\bra}[1]{\left\langle #1\right|}
\newcommand{\ket}[1]{\left|#1\right\rangle}
\newcommand{\braket}[2]{\left\langle #1\middle|#2\right\rangle}
\newcommand{\brakets}[3]{\left\langle #1\middle| #2 \middle|#3 \right\rangle}

\renewcommand{\arraystretch}{2.1}


\setlength{\textwidth}{16cm}
\setlength{\textheight}{25cm}
\setlength{\oddsidemargin}{0cm}
\setlength{\evensidemargin}{0cm}
\setlength{\topmargin}{-2cm}

\begin{document}
\maketitle

\vspace{1cm}
\begin{abstract}
    このノートは松原隆彦の``現代宇宙論ー時空と物質の共進化ー''の第3章の3節をまとめたものである. 要点や個人
的な追記, 計算ノート的なまとめを行っているが, それらはすべて原書の内容を出発点としている. 参考程度に使って いただきたいが, このノートは私の勉強ノートであり, そのままの内容をそのまま鵜呑みにすると間違った
理解を招く 可能性があることをご了承ください. ぜひ原著を手に取り, その内容をご自身で確認していただくことを推奨します.  てへぺろ v$({\hat{\cdot}_\partial \hat{\cdot}})$
v                                                          +
+
+
\end{abstract}
    
    

\newpage

\color{blue}
\subsection*{概要}
\begin{itemize}

    \item 宇宙論パラメータ:観測と理論の両面から宇宙の性質を特徴付ける物理量.
    \item 一様等方宇宙モデル(FLRW宇宙)では, 以下の量が宇宙論パラメータとして用いられる.
  
  \end{itemize}
  
  \subsection*{ハッブル定数 $H_0$}
  
  \begin{equation*}
    H_0 = \left. \frac{\dot{a}}{a} \right|_{t = t_0}
  \end{equation*}
  
  \begin{equation*}
    H_0 = 100\, h\, \text{km/s/Mpc} = 3.241 \times 10^{-18} h\, \text{s}^{-1}
  \end{equation*}
  
  \subsection*{臨界密度 $\varrho_{c0}$}
  
  \begin{equation*}
    \varrho_{c0} = \frac{3 H_0^2}{8 \pi G}
  \end{equation*}
  
  \begin{equation*}
    \varrho_{c0} \approx 1.878 \times 10^{-29} h^2\, \text{g/cm}^3
  \end{equation*}
  
  \subsection*{密度パラメータ $\Omega_A$}
  
  各成分 $A$ に対して,
  
  \begin{align*}
    \Omega_{A0} &= \frac{\rho_{A0}}{\rho_{c0}} \\
    \Omega_0 &= \sum_A \Omega_{A0}
  \end{align*}
  
  \subsection*{曲率パラメータ $\Omega_K$}
  
  \begin{equation*}
    \Omega_{K0} = -\frac{c^2 K}{H_0^2 a_0^2}
  \end{equation*}
  
  \begin{equation*}
    \Omega_0 + \Omega_{K0} = 1
  \end{equation*}
  
  \subsection*{減速パラメータ $q_0$}
  
  \begin{equation*}
    q_0 = -\frac{a \ddot{a}}{\dot{a}^2}
  \end{equation*}
  
  状態方程式パラメータ $w_A$ を用いると,
  
  \begin{equation*}
    q_0 = \frac{1}{2} \sum_A (1 + 3w_A) \Omega_{A0}
  \end{equation*}
  
  \subsection*{時間依存する宇宙論パラメータ}
  
  時間依存するハッブル関数は,
  
  \begin{equation*}
    H(t) = \frac{\dot{a}}{a}
  \end{equation*}
  
  このとき, 密度パラメータと曲率パラメータは次のように時間に依存する:
  
  \begin{align*}
    \Omega_A &= \frac{H_0^2}{H^2} \frac{\rho_A}{\rho_{A0}} \Omega_{A0} \\
    \Omega_K &= \frac{H_0^2}{H^2} \frac{\Omega_{K0}}{a^2}
  \end{align*}
  
  フリードマン方程式の時間発展形は,
  
  \begin{equation*}
    \frac{H^2}{H_0^2} = \sum_A \Omega_{A0} \frac{\rho_A}{\rho_{A0}} + \frac{\Omega_{K0}}{a^2}
  \end{equation*}
  
  \subsection*{応用例:複数成分の宇宙}
  
  物質、放射、宇宙定数など複数の成分からなる宇宙では, それぞれのエネルギー密度は
  
  \begin{align*}
    \rho_m &\propto a^{-3} \\
    \rho_r &\propto a^{-4} \\
    \rho_\Lambda &= \text{const}
  \end{align*}
  
  これにより,
  
  \begin{equation*}
    \Omega = \frac{\Omega_r a^{-4} + \Omega_m a^{-3} + \Omega_\Lambda}{\Omega_r a^{-4} + \Omega_m a^{-3} + \Omega_\Lambda + \Omega_K a^{-2}}
  \end{equation*}
  
  のように表される.
\color{black}

\section*{3.3. 宇宙論パラメータ}
宇宙の進化を解明するには, 理論だけではなく観測に基づいた「宇宙論パラメータ(cosmological parameters)」が
重要である. 特に一様等方宇宙モデルでは, Friedmann方程式を解くために必要なパラメータとして曲率 $K$ と全エネ ルギー密度 $\rho$ が必要となる. これらのパラメータは, 宇宙の進化を記述するた
めに重要である. 多様なパラメー タを組み合わせて宇宙論モデルを構築できるが, 本節では代表的かつ実践的なパラメータに絞って扱う.              +\subsection*{3.3.1 Hubble 定
数}
+Hubble 定数 $H_0$ は, 現在の宇宙の膨張率を決めるもので,
+\begin{equation*}
+    H_0 = \left. \frac{\dot{a}}{a} \right|_{t = t_0} \tag{3.53}
+\end{equation*}
+と定義される. Friedmann方程式で $t = t_0$ とすれば,
+\begin{equation*}
+    H_0^2 = \frac{8 \pi G}{3c^2} \rho_{0} -c^2 K. \tag{3.54}
+\end{equation*}
+ここで $\rho_{0}$ は現在の宇宙の全エネルギー密度である. 多成分の場合は,
+\begin{equation*}
+    \rho_{0} = \sum_{A} \rho_{A0}. \tag{3.55}
+\end{equation*}
+Hubble 定数は曲率と現在のエネルギー密度で与えられるので, 宇宙論パラメータである.\\
+規格化された量 $h$ を用いると,
+\begin{equation*}
+    H_0 = 100\, h\, \text{km/s/Mpc} = 3.241 \times 10^{-18} h\, \text{s}^{-1}. \tag{3.56}
+\end{equation*}
+\subsection*{3.3.2 臨界密度}
+平坦宇宙 ($K = 0$) での現在の全エネルギー密度 $\rho_{c0}$ は,
+\begin{equation*}
+    \rho_{c0} = \frac{3 c^2H_0^2}{8 \pi G} \tag{3.57}
+\end{equation*}
+と表される. これは, 臨界エネルギー密度と呼ばれ, これに対応する質量密度 $\varrho_{c0}$ は,
+\begin{equation*}
+    \varrho_{c0} = \frac{\rho_{c0}}{c^2} \tag{3.58}
+\end{equation*}
+となる. これも宇宙論パラメータで, その値は
+\begin{equation*}
+    \varrho_{c0} \approx 1.878 \times 10^{-29} h^2\, \text{g/cm}^3. \tag{3.59}
+\end{equation*}
+\subsection*{3.3.3 密度パラメータ}
+宇宙のエネルギー密度を臨界エネルギー密度で規格化した無次元量を密度パラメータという. 成分 $A$ に対して,
+\begin{equation*}
+    \Omega_{A0} = \frac{\rho_{A0}}{\rho_{c0}} = \frac{8\pi G\rho_{A0}}{3c^2 H_0^2} \tag{3.60}
+\end{equation*}
+と定義される. 全エネルギー密度 $\rho_{0}$ は,
+\begin{equation*}
+    \rho_{0} = \frac{\rho_{0}}{\rho_{c0}} = \frac{8\pi G\rho_{0}}{3c^2 H_0^2} = \sum_{A} \Omega_{A0} \tag{3.61}
+\end{equation*}
+となる. 現在の宇宙のエネルギー密度は放射成分をほぼ無視できるので,
+\begin{equation*}
+    \Omega_0 = \Omega_{m0} +  \Omega_{d0}. \tag{3.62}
+\end{equation*}
+ここで $\Omega_{m0}$ は物質成分の密度パラメータ, $\Omega_{d0}$ は暗黒物質成分の密度パラメータである. 

\end{document} 