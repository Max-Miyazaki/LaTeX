\documentclass[12pt]{jsarticle}

\title{Part 1 Special Relativity\\
1. The Geometry of Special Relativity}
\date{各種SNS\\
    X (旧 Twitter): \href{https://x.com/miya_max_study}{@miya\_max\_study}\\
    Instagram : \href{https://www.instagram.com/daily_life_of_miya/}{@daily\_life\_of\_miya}\\
    YouTube : \href{https://www.youtube.com/@miya-max-active}{@miya-max-active}
    }
\author{Max Miyazaki}

\usepackage{amsmath}
\usepackage{amssymb}
\usepackage{ascmac}
\usepackage{amsfonts}
\usepackage{color}
\usepackage[dvipdfmx]{graphicx}
\usepackage{float}
\usepackage{bm}
\usepackage{tcolorbox}
\usepackage{tikz}
\usetikzlibrary{decorations.markings}
\usepackage{indentfirst}
\usepackage[dvipdfmx]{hyperref}
\hypersetup{%
 setpagesize=false,
 bookmarksnumbered=true,
 colorlinks=true,
 linkcolor=blue}

% Define braket-like commands
\newcommand{\bra}[1]{\left\langle #1\right|}
\newcommand{\ket}[1]{\left|#1\right\rangle}
\newcommand{\braket}[2]{\left\langle #1\middle|#2\right\rangle}
\newcommand{\brakets}[3]{\left\langle #1\middle| #2 \middle|#3 \right\rangle}

\newcommand{\tcb}[2]{\begin{tcolorbox}[title={\textcolor{white}{#1}}, opacitybacktitle = 0, colframe=white!40!black]#2
\end{tcolorbox}}



\renewcommand{\arraystretch}{2.1}

\numberwithin{equation}{section}


\begin{document}
\maketitle
\vspace{1cm}
\begin{abstract}
    これは SIDNEY COLEMAN'S LECTURES ON RELATIVITY を各章でまとめたものです. メモ書き程度に色々追記がありますが、計算等に
\end{abstract}

\newpage
\tableofcontents
\newpage
\section{特殊相対性理論の幾何学}
\subsection{古典的物理系}
古典的な物理系は3つの部分で構成されている. 
\begin{enumerate}
    \item \textbf{四次元時空間}:古典物理学の舞台.時空間の点(\textbf{事象})を座標で表す:
    \begin{equation*}
    x^{\mu} = (x^0, x^i) = (ct, \mathbf{x}), \tag{1.1}
    \end{equation*}
    ここで $x^0$ は時間を表し($c = 1$ となる単位系を使う), $\mathbf{x}$ は位置を表す.ギリシャ文字のインデックス($\lambda, \mu, \nu, \ldots$) は $0$ から $3$ までの値を, ローマ字のインデックス ($i, j, k, \ldots$) は $1$ から $3$ までの値をとる.
    
    \item \textbf{粒子と場}:古典物理学の実体.
    
    \begin{enumerate}
    \item \textbf{粒子}: 粒子は構造を持たない点状物体である.時間の関数としての粒子の\textit{位置}$\mathbf{x}(t)$は, 粒子について言及できることすべてを教えてくれます(質量や電荷などの固定された特性を除いて). 4元ベクトル表記では, 粒子の軌跡(\textbf{世界線})を $x^{\mu}(s)$ で表します.ここで $s$ は曲線に沿った点を表すために使われるパラメータ(任意の単調関数 $f$ に対して $f(s)$ でも同様に機能する):
    \item \textbf{場}:
    \end{enumerate}
    \end{enumerate}



\end{document}