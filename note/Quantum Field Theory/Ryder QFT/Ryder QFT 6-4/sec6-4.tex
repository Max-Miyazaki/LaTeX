\documentclass{jsarticle}

\title{Ryder QFTゼミ資料}
\date{\today}
\author{Max Miyazaki}

\usepackage{amsmath}
\usepackage{amssymb}
\usepackage{ascmac}
\usepackage{amsfonts}
\usepackage{color}
\usepackage[dvipdfmx]{graphicx}
\usepackage{float}
\usepackage{bm}


% Define braket-like commands
\newcommand{\bra}[1]{\left\langle #1\right|}
\newcommand{\ket}[1]{\left|#1\right\rangle}
\newcommand{\braket}[2]{\left\langle #1\middle|#2\right\rangle}
\newcommand{\brakets}[3]{\left\langle #1\middle| #2 \middle|#3 \right\rangle}


\begin{document}
\maketitle
\section*{\textrm{6章 経路積分量子化とファインマンルール}}
\subsection*{\textrm{6.4 相互作用する場の生成関数}}

ラグランジアンは自身と相互作用するスカラー場を記述する.
\begin{equation*}
    \mathcal{L} = \frac{1}{2}\partial_\mu \phi \partial^\mu -\frac{1}{2}m^2 \phi^2 -\frac{g}{4!}\phi^4 = \mathcal{L}_0 + \mathcal{L}_\textrm{int} \tag{6.65}
\end{equation*}
まず一般的な相互作用 $\mathcal{L}_\textrm{int}$ に対するグリーン関数の求め方を示し, 次節でその公式を $\phi^4$ 理論に適応させる.\\
正規化された生成関数は,
\begin{equation}
    Z[J] = \frac{\displaystyle\int \mathcal{D}\phi \exp\left( iS + i\int J \phi dx \right)}{\displaystyle \int \mathcal{D}\phi e^{iS}} \tag{6.66}
\end{equation}
$\mathcal{L}_\textrm{int} = 0$ のとき, これが式 (6.43) となり, 式 (6.44) と同じであることを示すことができたのは明らか. 式 (6.44) は $J$ に対して汎関数微分するのに適した形であるのでグリーン関数を求めるのに適している. 相互作用する場での式 (6.44) に対応する式を求めたい. $Z[J]$ が満たす微分方程式を求めて, それを $Z_0[J]$ で解く.\\
まず式 (6.44) から
\begin{equation*}
    \frac{1}{i}\frac{\delta}{\delta J(x)} Z_o[J] = -\int \Delta_\textrm{F}(x-y)J(y)dy \exp\left( -\frac{i}{2}\int J\Delta_\textrm{F}J dxdy \right),
\end{equation*}
$\Delta_\textrm{F}$ は $\square + m^2$ の逆数を引くので,
\begin{equation*}
    (\square + m^2)\frac{1}{i}\frac{\delta}{\delta J(x)}Z_0[J] = J(x)Z_0[J] \tag{6.67}
\end{equation*}
これは $Z_0[J]$ が満たす微分方程式である.\par
ここで式 (6.66) から,
\begin{equation*}
    \frac{1}{i}\frac{\delta Z[J]}{\delta J(x)} = \frac{\displaystyle \int \exp\left( iS + i\int J\phi dx \right)\phi(x)\mathcal{D}\phi}{\displaystyle \int e^{iS}\mathcal{D}\phi} \tag{6.68}
\end{equation*}
汎関数を以下で定義する.
\begin{equation*}
    \hat{Z}[\phi] = \frac{e^{iS}}{\displaystyle \int e^{iS}\mathcal{D}\phi} \tag{6.69}
\end{equation*}
すると,
\begin{equation*}
    Z[J] = \int \hat{Z}[\phi]\exp\left[ i\int J(x)\phi(x) dx \right] \mathcal{D}\phi \tag{6.70}
\end{equation*}
これはフーリエ変換の汎関数的なアナロジーである. ここで
\begin{equation*}
    S = \int \left(\frac{1}{2}\partial_\mu \phi \partial^\mu -\frac{1}{2}m^2 \phi^2 -\frac{g}{4!}\phi^4 = \mathcal{L}_0 + \mathcal{L}_\textrm{int}\right) d^4 x = -\int \left[ \frac{1}{2}\phi (\square + m^2)\phi - \mathcal{L}_\textrm{int} \right] d^4 x \tag{6.71}
\end{equation*}
に注意しながら $\hat{Z}[\phi]$ の汎関数微分を取ると, 次を得る.
\begin{align*}
    i\frac{\delta \hat{Z}[\phi]}{\delta \phi(x)} &= i\frac{\delta}{\delta \phi}\left\{ \exp\left[ -i\int \left[ \frac{1}{2}\phi (\square + m^2)\phi - \mathcal{L}_\textrm{int} \right]d^4 x \right] \right\}\left[ \int e^{iS}\mathcal{D}\phi \right]^{-1}\\
    &= (\square + m^2)\phi \hat{Z}[\phi] -\frac{\partial \mathcal{L}_\textrm{int}}{\partial \phi}\hat{Z}[\phi]\\
    &= (\square + m^2)\phi \hat{Z}[\phi] - \mathcal{L}'_\textrm{int}\hat{Z}[\phi] \tag{6.72}
\end{align*}
ここで $\mathcal{L}'_\textrm{int} = \displaystyle \frac{\partial \mathcal{L}_\textrm{int}}{\partial \phi}$ である. 式 (6.72) の両辺に $\displaystyle \exp\left[ i\int J(x)\phi(x)dx \right]$ をかけて $\phi$ で積分する. 右辺は式 (6.68) が使われていた
\begin{align*}
    \frac{\displaystyle \int (\square + m^2)\phi \exp\left( iS + i\int J\phi dx \right)\mathcal{D}\phi}{\displaystyle \int e^{iS}\mathcal{D}\phi} -& \frac{\displaystyle \int \mathcal{L}'_{\textrm{int}}(\phi) \exp\left( iS + i\int J\phi dx \right)\mathcal{D}\phi}{\displaystyle \int e^{iS}\mathcal{D}\phi} \\
    &= (\square + m^2)\frac{1}{i}\frac{\delta Z[J]}{\delta J(x)} - \mathcal{L}'_{\textrm{int}}\left[ \frac{1}{i}\frac{\delta}{\delta J} \right]Z[J] \tag{6.73}
\end{align*}
を与え, $\mathcal{L}'_{\textrm{int}}$ の引数は $\phi$ から $\displaystyle \frac{1}{i}\frac{\delta}{\delta J}$ に変更されている. これは $Z[J]$ に作用するためである. 式 (6.72) の左辺は式 (6.70) から以下を与える.
\begin{equation*}
    (\square + m^2)\frac{1}{i}\frac{\delta Z[J]}{\delta J(x)} - \mathcal{L}'_{\textrm{int}}\left( \frac{1}{i}\frac{\delta}{\delta J(x)} \right)Z[J] = J(x)Z[J] \tag{6.75}
\end{equation*}
我々は $Z[J]$ の方程式を解かなければならない.\\
$\mathcal{L}'_{\textrm{int}} = 0$ の自由場では, 方程式は $Z[J]$ に対して式 (6.67) 式に帰着する. ここで式 (6.75) の解は
\begin{equation*}
    Z[J] = N \exp\left[ i\int \mathcal{L}'_{\textrm{int}}\left( \frac{1}{i}\frac{\delta}{\delta J} \right)dx \right]Z_0[J] \tag{6.76}
\end{equation*}
$N$ は正規化ファクターで, 証明は2つ段階がある.\par
証明\\
(a) 最初にこの恒等式を証明する.
\begin{equation*}
    \exp\left[ -i\int \mathcal{L}_{\textrm{int}} \left( \frac{1}{i}\frac{\delta}{\delta J(y)} \right)dy \right]J(x) \exp\left[ i\int \mathcal{L}_{\textrm{int}} \left( \frac{1}{i}\frac{\delta}{\delta J(y)} \right)dy \right] = J(x) - \mathcal{L}'_{\textrm{int}}\left( \frac{1}{i}\frac{\delta}{\delta J(x)} \right) \tag{6.77}
\end{equation*}
このことは,
\begin{equation*}
    \left[ x_i, \frac{1}{i}\frac{\partial}{\partial x_j} \right] = i\delta_{ij}
\end{equation*}
の汎関数的アナロジーが
\begin{equation*}
    \left[ J(x), \frac{1}{i}\frac{\delta}{\delta J(y)} \right] = i\delta (x-y)
\end{equation*}
であることを観察することによって導かれる.\\
この式を繰り返し適応すると次のようになる.
\begin{align*}
    \left[ J(x), \left( \frac{1}{i}\frac{\delta}{\delta J(y)} \right)^n \right] &= i\delta(x-y)\left( \frac{1}{i}\frac{\delta}{\delta J(y)} \right)^{n-1} + \frac{1}{i}\frac{\delta}{\delta J(y)}\left[ J(x), \left( \frac{1}{i}\frac{\delta}{\delta J(y)} \right)^{n-1} \right]\\
    &\hspace{0.3cm}\vdots\\
    &=  i\delta(x-y)n\left( \frac{1}{i}\frac{\delta}{\delta J(y)} \right)^{n-1} \tag{6.78}
\end{align*}
関数を
\begin{equation*}
    F(\phi) = F(0) + \phi F'(0) + \frac{\phi^2}{2!}F''(0) + \cdots = \sum_{n=0}^{\infty}\frac{\phi^n}{n!}F^{(n)}(0)
\end{equation*}
と展開し, $\phi \to \displaystyle \frac{1}{i}\frac{\delta}{\delta J}$ と置換することで式 (6.78) から以下が得られる.
\begin{equation*}
    \left[ J(x), \int \left( \frac{1}{i}\frac{\delta}{\delta J(y)} \right)dy \right] = iF'\left( \frac{1}{i}\frac{\delta}{\delta J(x)} \right) \tag{6.79}
\end{equation*}
ここで $A$, $B$ を演算子とし, $\displaystyle A = -i\int \mathcal{L}_\textrm{int}\frac{1}{i}\frac{\delta}{\delta J(y)}dy$, $B = J(x)$ とする Hausdorff の公式を使う.
\begin{equation*}
    e^A B e^{-A} = B + [A, B] + \frac{1}{2!}[A, [A, B]] + \cdots \tag{6.80}
\end{equation*}
この場合, $A$ は (式 (6.79) より) $[A, B]$ と交換するので, 式 (6.80) の右辺の最初の2項のみ現れて式 (6.77) が証明される.\\
\vskip\baselineskip
(b) ここで式 (6.76) が式 (6.75) の解であることを示さないといけない. 式 (6.76), 式 (6.77) から
\begin{align*}
    J(x)Z[J] &= \exp\left[ i\int \mathcal{L}_{\textrm{int}}\left( \frac{1}{i}\frac{\delta}{\delta J(y)} \right)dy \right]Z_0[J]\\
    &= N \exp\left[ i\int \mathcal{L}_{\textrm{int}}\left( \frac{1}{i}\frac{\delta}{\delta J(y)} \right)dy \right] \left[ J(x) - \mathcal{L}_{\textrm{int}}\left( \frac{1}{i}\frac{\delta}{\delta J(x)} \right) \right]Z_0[J]
\end{align*}
これらの項と初項は式 (6.67) を使って変換され, 2番目の項では $\displaystyle \exp\left[i\int \mathcal{L}_{\textrm{int}}\right]$ と $\mathcal{L}'_{\textrm{int}}$ の順序を入れ替えることで, 式 (6.76) を用いた以下の式を与えることができる.
\begin{align*}
    J(x)Z[J] &= N\exp\left[ i\int \mathcal{L}_{\textrm{int}}\left( \frac{1}{i}\frac{\delta}{\delta J(y)} \right)dy \right](\square + m^2)\frac{1}{i}\frac{\delta Z_0}{\delta J(x)} - N\mathcal{L}'_{\textrm{int}}\left( \frac{1}{i}\frac{\delta}{\delta J(x)} \right)\exp\left[ i\int \mathcal{L}_{\textrm{int}}\left( \frac{1}{i}\frac{\delta}{\delta J(y)} \right)dy \right]Z_0[J]\\
    &= (\square + m^2)\frac{\delta Z_0[J]}{\delta J(x)} - \mathcal{L}'_{\textrm{int}}\frac{1}{i}\left[ \frac{\delta}{\delta J(x)} \right]Z[J]
\end{align*}
これが式 (6.75) である. 証明完了.\par
これで相互作用する場でのグリーン関数を計算できるようになった. これは量子論で普通行われるように摂動論によって計算される.




\end{document}