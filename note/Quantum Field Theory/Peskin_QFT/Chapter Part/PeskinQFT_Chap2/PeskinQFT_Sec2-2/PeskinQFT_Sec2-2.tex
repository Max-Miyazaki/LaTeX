\documentclass[a4paper,12pt]{article}

\title{Chapter 2. The Klein-Gordon Field\\
2-2. Elements of Classical Field Theory}
\date{各種SNS\\
    X (旧 Twitter): \href{https://x.com/miya_max_study}{@miya\_max\_study}\\
    Instagram : \href{https://www.instagram.com/daily_life_of_miya/}{@daily\_life\_of\_miya}\\
    YouTube : \href{https://www.youtube.com/@miya-max-active}{@miya-max-active}
    }
\author{Max Miyazaki}

\usepackage{amsmath}
\usepackage{amssymb}
\usepackage{ascmac}
\usepackage{amsthm}
\usepackage{amsfonts}
\usepackage{enumitem}
\usepackage{color}
\usepackage[dvipdfmx]{graphicx}
\usepackage{float}
\usepackage{bm}
\usepackage{here}

\usepackage{abstract}
\usepackage{tikz}
\usetikzlibrary{shapes.geometric, arrows.meta, positioning}
\usepackage{indentfirst}
\usepackage[utf8]{inputenc}
\usepackage{fix-cm}
\usepackage{wrapfig}
\pagenumbering{arabic}
\usepackage{url}
\usepackage{xcolor}
\usepackage[most]{tcolorbox}
\usepackage{framed}
\usepackage[dvipdfmx]{hyperref}
\hypersetup{
 setpagesize=false,
 bookmarksnumbered=true,
 colorlinks=true,
 linkcolor=blue
}

% Define braket-like commands
\newcommand{\bra}[1]{\left\langle #1\right|}
\newcommand{\ket}[1]{\left|#1\right\rangle}
\newcommand{\braket}[2]{\left\langle #1\middle|#2\right\rangle}
\newcommand{\brakets}[3]{\left\langle #1\middle| #2 \middle|#3 \right\rangle}

\renewcommand{\arraystretch}{2.1}


\setlength{\textwidth}{16cm}
\setlength{\textheight}{25cm}
\setlength{\oddsidemargin}{0cm}
\setlength{\evensidemargin}{0cm}
\setlength{\topmargin}{-2cm}

\begin{document}
\maketitle

\vspace{1cm}
\begin{abstract}
    このノートはPeskin\&Schroederの``An Introduction to Quantum Field Theory''の第2章の2節をまとめたものである. 要点や個人的な追記, 計算ノート的なまとめを行っているが, それらはすべて原書の内容を出発点としている. 参考程度に使っていただきたいが, このノートは私の勉強ノートであり, そのままの内容をそのまま鵜呑みにすると間違った理解を招く可能性があることをご了承ください. ぜひ原著を手に取り, その内容をご自身で確認していただくことを推奨します. てへぺろ v$({\hat{\cdot}_\partial \hat{\cdot}})$v



\end{abstract}
    
    

\newpage
\section*{2.2 Elements of Classical Field Theory}
\subsection*{Lagrangian Field Theory}

\color{blue}
※ここで登場する $\phi$ は物理空間(時空)の各点 $x$ に定義された『場(field)』である. 具体的には, 各点 $x^\mu = (t, \vec{x})$ に対してある値を取る関数 $\phi(x)$ のことを指す.

このような場は, 次のような観点で理解できる:

\begin{itemize}
    \item $\phi$ は空間や時間の各点において定義される変数で, 例えば温度分布や電場のように『空間全体に広がった物理量』を表現する.
    \item 本節では, $\phi$ を \textbf{スカラー場(scalar field)} として扱う. つまり, 空間や時空の点ごとに \emph{ただ一つの実数値} をとる場を考える.
    \item スカラー場の例としては, 宇宙背景放射の温度分布, あるいは場の量子論におけるヒッグス場などがある.
    \item より一般的にはベクトル場やスピノル場なども存在するが, 基本的な構造を理解するためにはスカラー場が最も扱いやすい.
\end{itemize}

したがって, この節では $\phi(x)$ を \textbf{実数値のスカラー場} (実スカラー場) として議論を進める. この仮定のもとで, Lagrangian 密度 $\mathcal{L}$ は $\phi(x)$ とその導関数 $\partial_\mu \phi(x)$ の関数として定義され, そこから作用 $S$ や運動方程式が導出される.

\color{black}

\begin{itemize}
    \item \textbf{作用} $S$: Lagrangian の時間積分として定義される.
    \begin{equation*}
        S = \int L\,dt = \int \mathcal{L}(\phi, \partial_\mu \phi)\,d^4x \tag{2.1}
    \end{equation*}

    \item \textbf{Lagrangian 密度} $\mathcal{L}$: 場 $\phi(x)$ およびその導関数 $\partial_\mu \phi$ の関数.\\
    \hspace{4cm} \textcolor{blue}{ (以降, この $\mathcal{L}$ を単に Lagrangian と呼ぶ.) }

    \item \textbf{最小作用の原理}:物理系は作用 $S$ が極値(通常は最小)をとる経路に沿って変化.
    
    \begin{align*}
        0 = \delta S 
        &= \int d^4x \left\{ \frac{\partial \mathcal{L}}{\partial \phi} \, \delta \phi 
        + \frac{\partial \mathcal{L}}{\partial (\partial_\mu \phi)} \, \delta(\partial_\mu \phi) \right\} \\
        &= \int d^4x \left\{ \frac{\partial \mathcal{L}}{\partial \phi} \, \delta \phi 
        - \partial_\mu \left( \frac{\partial \mathcal{L}}{\partial (\partial_\mu \phi)} \right) \delta \phi 
        + \partial_\mu \left( \frac{\partial \mathcal{L}}{\partial (\partial_\mu \phi)} \delta \phi \right) \right\}. \tag{2.2}
    \end{align*}
        

    \item \textbf{場に対する Euler-Lagrange 方程式}:
    \begin{equation}\label{2.3}
        \partial_\mu \left( \frac{\partial \mathcal{L}}{\partial (\partial_\mu \phi)} \right) - \frac{\partial \mathcal{L}}{\partial \phi} = 0 \tag{2.3}
    \end{equation}
\end{itemize}

\noindent Lagrangian に複数の場が含まれている場合, それぞれの場に対して同様の方程式が存在.

\color{blue}

\begin{proof}
※場の Euler-Lagrange 方程式を最小作用の原理から導出する.

\begin{align*}
    \delta S &= 0 \tag{2-2.a1} \\
    \Rightarrow \delta S &= \int d^4x \left\{ \underline{\mathcal{L}(\phi + \delta \phi, \partial_\mu \phi + \delta \partial_\mu \phi)} - \mathcal{L}(\phi, \partial_\mu \phi) \right\} \tag{2-2.a2} \\
    &\quad \hspace{1.7cm} \mathcal{L}(\phi, \partial_\mu \phi) 
    + \delta \phi \frac{\partial \mathcal{L}}{\partial \phi} 
    + \delta (\partial_\mu \phi) \frac{\partial \mathcal{L}}{\partial (\partial_\mu \phi)} \tag{2-2.a3} \\
    &= \int d^4x \left\{ 
    \delta \phi \frac{\partial \mathcal{L}}{\partial \phi} 
    + \underline{\delta (\partial_\mu \phi) \frac{\partial \mathcal{L}}{\partial (\partial_\mu \phi)}} 
    \right\} \tag{2-2.a4} \\
    &\quad \hspace{1cm} \delta(\partial_\mu \phi)\frac{\partial \mathcal{L}}{\partial(\partial_\mu \phi)}+  \delta \phi \partial_\mu \left( \frac{\partial \mathcal{L}}{\partial (\partial_\mu \phi)} \right) - \delta \phi \partial_\mu \left( \frac{\partial \mathcal{L}}{\partial (\partial_\mu \phi)} \right) \tag{2-2.a5} \\
    &\quad \hspace{1cm} \Leftrightarrow - \delta \phi \partial_\mu \left( \frac{\partial \mathcal{L}}{\partial (\partial_\mu \phi)} \right) + \left\{ \underline{\delta(\partial_\mu \phi)}\, \frac{\partial \mathcal{L}}{\partial(\partial_\mu \phi)} + \delta \phi \partial_\mu \left( \frac{\partial \mathcal{L}}{\partial (\partial_\mu \phi)} \right) \right\} \tag{2-2.a6} \\
    & \quad \hspace{4.7cm} \delta (\partial_\mu \phi) = \partial_\mu (\delta \phi) \tag{2-2.a7} \\
    & \quad \hspace{1cm} \Leftrightarrow - \delta \phi \partial_\mu \left( \frac{\partial \mathcal{L}}{\partial (\partial_\mu \phi)} \right) + \left\{ \partial_\mu(\delta \phi)\, \frac{\partial \mathcal{L}}{\partial(\partial_\mu \phi)} + \delta \phi \partial_\mu \left( \frac{\partial \mathcal{L}}{\partial (\partial_\mu \phi)} \right) \right\} \tag{2-2.a8} \\
    & \quad \hspace{1cm} \Leftrightarrow - \delta \phi \partial_\mu \left( \frac{\partial \mathcal{L}}{\partial (\partial_\mu \phi)} \right) + \partial_\mu \left( \delta \phi \frac{\partial \mathcal{L}}{\partial (\partial_\mu \phi)} \right) \tag{2-2.a9} \\
    &= \int d^4x \left\{ 
    \delta \phi \frac{\partial \mathcal{L}}{\partial \phi} 
    - \delta \phi \partial_\mu \left( \frac{\partial \mathcal{L}}{\partial (\partial_\mu \phi)} \right)
    + \underline{\partial_\mu \left( \delta \phi \frac{\partial \mathcal{L}}{\partial (\partial_\mu \phi)} \right)}
    \right\} \tag{2-2.a10} \\
    &\quad \hspace{6.5cm} \to 0 \hspace{0.5cm}(\because \text{境界条件}\delta \phi_I = \delta \phi_F = 0) \\
    &= \int d^4x  
    \left[ \frac{\partial \mathcal{L}}{\partial \phi} 
    - \partial_\mu \left( \frac{\partial \mathcal{L}}{\partial (\partial_\mu \phi)} \right)
    \right] \delta \phi. \tag{2-2.a11}
\end{align*}

この積分が任意の $\delta \phi$ に対してゼロであるためには括弧内がゼロである必要がある.\\
以上により, 場に対する Euler-Lagrange 方程式が得られる.

\begin{equation*}
    \partial_\mu \left( \frac{\partial \mathcal{L}}{\partial (\partial_\mu \phi)} \right) - \frac{\partial \mathcal{L}}{\partial \phi} = 0 \tag{2-2.a12}
\end{equation*}

\end{proof}

\color{black}

場の運動は作用 $S$ が場の小さな変分に対して不変(すなわち極値を取る)となるように決定される. この条件から Euler-Lagrange 方程式が導かれる.

\subsection*{Hamiltonian Field Theory}

Lagrangian 形式の場の理論は, すべての式が明示的に Lorentz 不変となっているため, 特に相対論的ダイナミクスに適している. しかし, 本書の前半では量子力学への移行を容易にするために Hamiltonian 形式を用いる.
\vskip\baselineskip

\color{blue}
※もう少し Hamiltonian 形式のモチベーションについて述べておく.\\
Hamiltonian 形式では時間を特別な変数として扱う. つまり, $\dot{\phi} = \partial_0 \phi$ を独立変数とするため, 空間と時間の対称的な扱いが失われてしまう. これは時空の対称性 (Lorentz不変性) を明示的に失うことを意味する. だが以下のような利点がある.

\begin{itemize}
    \item 正準量子化 (canonical quantization) がやりやすい.\par
    位置と運動量に対して Poisson 括弧や交換関係をを導入するには時間方向を取り出して『時間発展』の枠組みが必要となる. 
    \item 場の量子論での標準的計算手法に繋がる.\par
    Heisenberg 描像, Feynman 経路積分といった形式に自然に繋がる.
\end{itemize}


\noindent Question:え?でも Lorentz 不変性破ってそんなことして大丈夫なんですか?\\
Answer: Hamiltonian 形式では『一時的に』 Lorentz 不変性を『明示的』に捨てるが, 最終的には理論全体として Lorentz 不変性を回復させることができるので問題ない!

\vskip\baselineskip

理論物理の立場としては,\\
『一時的に非共変になっても良く, 最終的に物理的結果が共変ならOK』

\vskip\baselineskip

Lagrangian 形式も Hamiltonian 形式もそれぞれに利点があるため, 必要に応じてどちらの形式で議論するのか分けると良い.

\color{blue}
\begin{table}[h]
    \color{blue}
    \centering
    \renewcommand{\arraystretch}{1.5}
    \begin{tabular}{|p{7cm}|p{7cm}|}
    \hline
    \textbf{ Lagrangian 形式の利点} & \textbf{ Hamiltonian 形式の利点} \\
    \hline
    Lorentz 共変性が明示的 & 状態の時間発展が記述しやすい \\
    \hline
    対称性・ゲージ理論に対して自然 & 正準量子化に適している \\
    \hline
    経路積分による量子化ができる & 演算子形式に基づいた定式化ができる \\
    \hline
    \end{tabular}
    \caption{Lagrangian 形式と Hamiltonian 形式の利点の比較}
\end{table}

この理論は Lorentz 共変な Lagrangian $\mathcal{L}$ からスタートしているので, 理論の根本的な構造に Lorentz 対称性が組み込まれている. この辺の詳細な議論は \textit{Weinberg QFT Vol.1} や, 3.1 Lorentz Invariance in Wave Equations で言及するのでそちらを参照されたし.

\color{black}

\vskip\baselineskip
離散的な系では, 共役運動量 $p \equiv \partial L / \partial \dot{q}$ を各動的変数 $q$ に対して定義可能. このときハミルトニアンは次のように定義される.

\begin{equation*}
    H \equiv \sum p \dot{q} - L, \hspace{0.5cm} \left(\dot{q} = \frac{\partial q}{\partial t}\right)
\end{equation*}

連続系への一般化は, 空間点 $x$ を離散的に離れてると考え, 次のように定義する.

\begin{align*}
    p(x) &\equiv \frac{\partial L}{\partial \dot{\phi}(x)}
    = \frac{\partial}{\partial \dot{\phi}(x)} \int \mathcal{L}(\phi(y), \dot{\phi}(y))\, d^3y \notag \\
    &\sim \frac{\partial}{\partial \dot{\phi}(x)} \sum_y \mathcal{L}(\phi(y), \dot{\phi}(y))\, d^3y \notag \\
    &= \pi(x) d^3x,
\end{align*}

ここで,

\begin{equation}
    \pi(x) \equiv \frac{\partial \mathcal{L}}{\partial \dot{\phi}(x)} \tag{2.4}
\end{equation}

は, $\phi(x)$ に共役な運動量密度(momentum density)と呼ばれる.\par
\color{blue}
※共役とは, ある物理量に対応して変化の影響や応答を定量化する対の量.\\
今回の場合, $\pi$ は時間方向の変化に対する Lagrangian の応答を測るという意味で $\phi(x)$ に共役な量 (共役な運動量密度) になる.

\color{black}

このとき, ハミルトニアンは

\begin{equation*}
    H = \sum_x p(x) \dot{\phi}(x) - L
\end{equation*}

と書ける. これを連続極限に移すと, 次のようになる:

\begin{equation}
    H = \int d^3x \left[ \pi(x) \dot{\phi}(x) - \mathcal{L} \right] 
    \equiv \int d^3x\, \mathcal{H}. \tag{2.5}
\end{equation}

このセクションの後半では, 別の方法を用いてこのハミルトニアン密度 $\mathcal{H}$ の表式を再導出する予定である.

簡単な例として, 単一の場 $\phi(x)$ によって記述される理論を考える. そのときのラグランジアンは以下の通り:

\begin{align}\label{2.6}
    \mathcal{L} &= \frac{1}{2} \dot{\phi}^2 - \frac{1}{2} (\nabla \phi)^2 - \frac{1}{2} m^2 \phi^2 \notag \\
    &= \frac{1}{2} (\partial_\mu \phi)^2 - \frac{1}{2} m^2 \phi^2. \tag{2.6}
\end{align}

$\phi$:実スカラー場とする. $m$:パラメータ (§2.3 において質量として解釈される).\\
このラグランジアンから運動方程式が得られる:

\begin{equation*}\label{2.7}
\left( \frac{\partial^2}{\partial t^2} - \nabla^2 + m^2 \right) \phi = 0
\quad \text{または} \quad
\left( \partial^\mu \partial_\mu + m^2 \right) \phi = 0, \tag{2.7}
\end{equation*}

\color{blue}

\begin{proof}
\eqref{2.7}式の導出.\\
Euler-Lagrange 方程式 \eqref{2.3} に \eqref{2.6} の Lagrangian を入れて計算する.
\begin{align*}
    \partial_\mu \left( \frac{\partial \mathcal{L}}{\partial (\partial_\mu \phi)} \right) - \frac{\partial \mathcal{L}}{\partial \phi} &= 0 \tag{2-2.b1}\\
    \partial_\mu \left( \frac{\partial}{\partial (\partial_\mu \phi)} \left\{ \frac{1}{2} (\partial_\mu \phi)^2 - \frac{1}{2} m^2 \phi^2 \right\} \right) - \frac{\partial}{\partial \phi} \left( \frac{1}{2} (\partial_\mu \phi)^2 - \frac{1}{2} m^2 \phi^2 \right) &= 0 \tag{2-2.b2}\\
    \partial_\mu \left( \frac{\partial}{\partial (\partial_\mu \phi)} \left\{ \frac{1}{2} g^{\mu\nu}(\partial_\mu \phi)(\partial_\nu \phi) - \frac{1}{2} m^2 \phi^2 \right\} \right) - \frac{\partial}{\partial \phi} \left( \frac{1}{2} (\partial_\mu \phi)^2 - \frac{1}{2} m^2 \phi^2 \right) &= 0 \tag{2-2.b2}\\
    \partial_\mu \left( g^{\mu\nu}(\partial_\nu \phi) \right) - m^2 \phi &= 0 \tag{2-2.b3}\\
    (\partial_\mu \partial^\mu -m^2) \phi &= 0 \tag{2-2.b4}
\end{align*}

\end{proof}

\color{black}

これはよく知られた Klein-Gordon 方程式.(この文脈では、これはMaxwell 方程式のような\textbf{古典的}場の方程式であり、量子力学的な波動方程式ではない). $\phi(x)$ に共役な正準運動量密度が $\pi(x) = \dot{\phi}(x)$ であることに注意する.
\vskip\baselineskip

\color{blue}

※急に $\phi$ の名称に『\textbf{正準}』共役運動量と変わったが, 正準とは一般的に『基本的な形に従っている』『標準的な構造を持つ』という意味で使われる.

\begin{itemize}
    \item 共役運動量の定義
    スカラー場 $\phi$ に対して, 共役運動量 $\pi$ は次のように定義される:
    \begin{equation*}
        \pi = \frac{\partial \mathcal{L}}{\partial \dot{\phi}(x)}. \tag{2-2.c1}
    \end{equation*}
    これは単なる『共役運動量』である.
    \item 正準共役運動量とは\\
    $(\phi(x), \pi(x))$ が Poisson 括弧 $\{\phi(x), \pi(y)\} = \delta^3(x-y)$, 量子化後なら交換関係 $[\hat{\phi}(x), \hat{\pi}(y)] = i\delta^3(x-y)$ を満たすとき, $\pi(x)$ は $\phi(x)$ に対して正準共役であるという. わざわざ『正準』とつける理由は, 共役運動量を一般的に定義できても正準構造を満たさない場合がある(ゲージ場理論での拘束系など).
\end{itemize}

\color{black}

そうすれば Hamiltonian も構成できる:

\begin{equation*}\label{2.8}
H = \int d^3x\, \mathcal{H} = \int d^3x \left[ \frac{1}{2} \pi^2 + \frac{1}{2} (\nabla \phi)^2 + \frac{1}{2} m^2 \phi^2 \right]. \tag{2.8}
\end{equation*}

この3つの項はそれぞれ「時間方向の運動」に伴うエネルギー, 「空間方向の\textbf{ずれ}(shearing)」に伴うエネルギー, そして場がそこに存在すること自体にかかるエネルギーと解釈できる. この Hamiltonian については、セクション 2.3 および 2.4 においてさらに詳しく考察する.

\section*{Noether's Theorem}

古典的な場の理論における対称性と保存則の関係について議論する(Noether の定理).\\
この定理は、場 $\phi$ に対する連続変換について述べており、この変換は無限小形では次のように表される:
\begin{equation}\label{2.9}
\phi(x) \to \phi'(x) = \phi(x) + \alpha \Delta \phi(x),
\tag{2.9}
\end{equation}
ここで,$\alpha$ は無限小パラメータ(小さな量), $\Delta \phi$ は場の配位のある変形を表す.\\
もし作用が \eqref{2.9} の変換に対して不変であるならば, 運動方程式が不変であるということで, この変換を対称変換という.
より一般的には, 作用が表面項(表面積分に帰着する項)だけ変わることを許す. $\Longrightarrow$ 表面項は Euler-Lagrange 方程式の導出に影響ないから.\\
したがって, Lagrangian $\mathcal{L}$ は \eqref{2.9} のもとで, 4次元発散 (four-divergence) を除いて不変でなくてはならない:
\begin{equation*}\label{2.10}
\mathcal{L}(x) \to \mathcal{L}(x) + \alpha \partial_\mu \mathcal{J}^\mu(x),
\tag{2.10}
\end{equation*}
ここで,$\mathcal{J}^\mu$ は何らかの関数である(後で保存カレントとなるもの).
場を変化させたとき, Lagrangian の変化 $\alpha \Delta \mathcal{L}$ は以下のように計算される:
\begin{align*}
\alpha \Delta \mathcal{L} &= \frac{\partial \mathcal{L}}{\partial \phi} (\alpha \Delta \phi) + \left( \frac{\partial \mathcal{L}}{\partial (\partial_\mu \phi)} \right) \partial_\mu (\alpha \Delta \phi) \notag \\
&= \alpha \partial_\mu \left( \frac{\partial \mathcal{L}}{\partial (\partial_\mu \phi)} \Delta \phi \right) + \alpha \left[ \frac{\partial \mathcal{L}}{\partial \phi} - \partial_\mu \left( \frac{\partial \mathcal{L}}{\partial (\partial_\mu \phi)} \right) \right] \Delta \phi.
\label{2.11}\tag{2.11}
\end{align*}

\color{blue}

\begin{proof}\eqref{2.11}の導出.\\
$\mathcal{L}(\phi, \partial_\mu \phi) \to \mathcal{L}(\phi + \alpha\Delta\phi, \partial_\mu \phi + \alpha\Delta\partial_\mu \phi)$ より,
\begin{align*}
    \alpha\Delta\mathcal{L} &= \mathcal{L}(\phi + \alpha\Delta\phi, \partial_\mu \phi + \alpha\Delta\partial_\mu \phi) - \mathcal{L}(\phi, \partial_\mu \phi) \tag{2-2.d1}\\
    &= \mathcal{L}(\phi, \partial_\mu \phi) + \alpha\Delta\phi\frac{\partial \mathcal{L}}{\partial \phi} + \partial_\mu (\alpha\Delta\phi) \frac{\partial \mathcal{L}}{\partial (\partial_\mu \phi)} - \mathcal{L}(\phi, \partial_\mu \phi) \tag{2-2.d2}\\
    &= \alpha\Delta\phi\frac{\partial \mathcal{L}}{\partial \phi} + \alpha\underline{\partial_\mu (\Delta\phi) \frac{\partial \mathcal{L}}{\partial (\partial_\mu \phi)}} \tag{2-2.d3}\\
    &\quad\hspace{1cm} \partial_\mu (\Delta\phi) \frac{\partial \mathcal{L}}{\partial (\partial_\mu \phi)} = \underline{\partial_\mu (\Delta\phi)\frac{\partial \mathcal{L}}{\partial(\partial_\mu \phi)} + \Delta\phi\partial_\mu \frac{\partial \mathcal{L}}{\partial(\partial_\mu \phi)}} - \Delta\phi\partial_\mu \frac{\partial \mathcal{L}}{\partial(\partial_\mu \phi)} \tag{2-2.d4}\\
    &\quad\hspace{5.3cm} \partial_\mu \left( \Delta\phi \frac{\partial \mathcal{L}}{\partial(\partial_\mu \phi)}\right) \tag{2-2.d5}\\
    &= \alpha\Delta\phi\frac{\partial \mathcal{L}}{\partial \phi} + \alpha\left[ \partial_\mu \left( \Delta\phi \frac{\partial \mathcal{L}}{\partial(\partial_\mu \phi)}\right) - \Delta\phi\partial_\mu \frac{\partial \mathcal{L}}{\partial(\partial_\mu \phi)} \right] \tag{2-2.d6}\\
    &= \alpha \partial_\mu \left( \Delta\phi \frac{\partial \mathcal{L}}{\partial(\partial_\mu \phi)}\right) + \alpha\left[ \frac{\partial \mathcal{L}}{\partial \phi} - \partial_\mu \frac{\partial \mathcal{L}}{\partial(\partial_\mu \phi)} \right]\Delta\phi \tag{2-2.d7}
\end{align*}
\end{proof}

\color{black}

第2項は Euler-Lagrange 方程式 \eqref{2.3} により消える.\\
残りの項を $\alpha \partial_\mu \left( \Delta\phi \dfrac{\partial \mathcal{L}}{\partial(\partial_\mu \phi)}\right) \equiv \alpha \partial_\mu \mathcal{J}^\mu$ と置くと, 次の式を得る:
\begin{equation}\label{2.12}
\partial_\mu j^\mu(x) = 0, \quad \text{for} \quad j^\mu(x) = \frac{\partial \mathcal{L}}{\partial(\partial_\mu \phi)} \Delta \phi - \mathcal{J}^\mu.
\tag{2.12}
\end{equation}

\color{blue}

\begin{proof}
\eqref{2.12}の導出.\\
\begin{align*}
    \alpha \partial_\mu \left( \Delta\phi \dfrac{\partial \mathcal{L}}{\partial(\partial_\mu \phi)}\right) &= \alpha \partial_\mu \mathcal{J}^\mu \tag{2-2.e1}\\
    \partial_\mu \left( \Delta\phi \dfrac{\partial \mathcal{L}}{\partial(\partial_\mu \phi)}\right) &= \partial_\mu \mathcal{J}^\mu \tag{2-2.e2}\\
    \partial_\mu \left( \Delta\phi \dfrac{\partial \mathcal{L}}{\partial(\partial_\mu \phi)}\right) &= \partial_\mu \left( \frac{\partial \mathcal{L}}{\partial(\partial_\mu \phi)} \Delta \phi - j^\mu(x) \right) \tag{2-2.e3}\\
    \partial_\mu \left( \Delta\phi \dfrac{\partial \mathcal{L}}{\partial(\partial_\mu \phi)}\right) &= \partial_\mu \left( \frac{\partial \mathcal{L}}{\partial(\partial_\mu \phi)} \Delta \phi \right) - \partial_\mu j^\mu(x) \tag{2-2.e4}\\
    \therefore \partial_\mu j^\mu(x) &= 0 \tag{2-2.e5}
\end{align*}

\end{proof}

\color{black}

\noindent (もし対称変換が複数の場にまたがる場合, この式の最初の項は各場に対する和になる)

この結果は, カレント $j^\mu(x)$ が保存されることを示している. Lagrangian $\mathcal{L}$ に対して各連続対称性が存在するごとに, このような保存則が存在する.

保存則はまた,チャージ $Q$ を用いて次のように表現することもできる:
\begin{equation}\label{2.13}
Q \equiv \int_{\text{all space}} j^0 \, d^3x.
\tag{2.13}
\end{equation}
これは時間に対して一定(保存)である.
\vskip\baselineskip

\color{blue}

\noindent ※『チャージ』『カレント』とは?
\begin{itemize}
    \item チャージ:\\
    ある場や系に対して対称性が存在するとき, そこに対応する保存量が現れる. \\
    この保存量を\textbf{チャージ}という.
    \item カレント:\\
    チャージが空間に分布していて, その分布が時間によって変化するとき, その変化を表す量を\textbf{カレント}という.
\end{itemize}

\begin{proof}
チャージが時間に対して一定であることを示す.\\
保存則 \eqref{2.12} より,
\begin{align*}
    \partial_\mu j^\mu &= 0 \tag{2-2.f1}\\
    \frac{\partial j^0}{\partial t} - \nabla \cdot \boldsymbol{j} &= 0 \tag{2-2.f2}\\
    \frac{\partial j^0}{\partial t} &= \nabla \cdot \boldsymbol{j}. \tag{2-2.f3}
\end{align*}

チャージの時間変化を計算する.

\begin{align*}
    \frac{\partial Q}{\partial t} &= \int_{\text{all space}} \frac{\partial j^0}{\partial t} \, d^3x \tag{2-2.f4}\\
    &= \int_{\text{all space}} \nabla \cdot \boldsymbol{j} \, d^3x \tag{2-2.f5}\\
    &= \int_{\text{surface}} \boldsymbol{j} \cdot d\boldsymbol{S} \hspace{0.5cm} (\because \text{Gauss の定理}) \label{2-2.f6}\tag{2-2.f6}\\
    &= 0. \tag{2-2.f7}
\end{align*}

\eqref{2-2.f6} は表面積分となっており, 通常, 物理的に意味のある場 (ex. 電流密度, スカラー場など) については次の仮定をおく.
\begin{center}
    『場は無限遠で急速にゼロに近づく(少なくとも $1/r^2$ より速く減速する)』    
\end{center}
※場は物理的な「物質」や「エネルギー分布」を表している. それらは有限の領域に存在していて, 無限遠に行くとそれらの影響は感じなくなるのが普通である.\\
よって, 『まともな場』は無限遠ではゼロになるべきだと考えられる.
\vskip\baselineskip
ではなぜ $1/r^2$ より速くなのか?$\Longrightarrow$ 積分を考えるとき重要な基準となる.
\vskip\baselineskip
空間の体積積分を考えたとき, 例えば, 今回のようにカレント密度 $\boldsymbol{j}$ の発散 $\nabla \cdot \boldsymbol{j}$ を\\空間積分すると, Gauss の定理によって境界の表面積分に変換される.
\vskip\baselineskip
ここで問題となるのが, 無限遠の表面積の大きさである.
\vskip\baselineskip
3次元空間では半径 $r$ の球の表面積 $\boldsymbol{S}$ は, $\boldsymbol{S} = 4\pi r^2$.\\
つまり無限遠に行くと表面積は $r^2$ に比例して急激に大きくなる. \par
$\Longrightarrow$ $1/r^2$ より速く減衰してくれると, 無限遠で安全に表面積分がゼロになる.

\end{proof}

\color{black}

局所的な Lagrangian 密度を用いた場の理論の定式化が, この局所的な保存則の形 \eqref{2.12} を直接導くということ. つまり, 場の理論を局所的に組み立てると, 局所的な保存則が自然に出てくる.\par
最も簡単な保存則の例は, 次のような Lagrangian から生じる:

\begin{equation*}
\mathcal{L} = \frac{1}{2} (\partial_\mu \phi)^2.
\end{equation*}

この場合, 変換 $\phi \to \phi + \alpha$ ($\alpha$ は定数)に対して Lagrangian $\mathcal{L}$ は不変である. したがって, 電流 $j^\mu = \partial^\mu \phi$ が保存されると結論づけられる.\par
より非自明な例として, 次の Lagrangian を考える:

\begin{equation*}
\mathcal{L} = |\partial_\mu \phi|^2 - m^2 |\phi|^2,
\tag{2.14}
\end{equation*}

ここで $\phi$ は複素数値の場である.\\
この Lagrangian に対する運動方程式が Klein-Gordon 方程式 \eqref{2.7} である.\par
この Lagrangian は変換 $\phi \to e^{i\alpha} \phi$ に対して不変であり, 無限小変換では,

\begin{equation}
\alpha \Delta \phi = i\alpha \phi, \quad \alpha \Delta \phi^* = -i\alpha \phi^*.
\tag{2.15}
\end{equation}

\noindent(ここでは $\phi$ と $\phi^*$ を独立な場として扱う. または, $\phi$ の実部と虚部に分けて議論可能.)

この場合, 保存される Noether 電流は次のように求められる:

\begin{equation*}\label{2.16}
j^\mu = i \left[ (\partial^\mu \phi^*) \phi - \phi^* (\partial^\mu \phi) \right]. \tag{2.16}
\end{equation*}

この式における全体の定数は恣意的に選んでいる.

\color{blue}

\begin{proof}
\eqref{2.16}の導出.\par
まずは Lagrangian が $\phi \to e^{i\alpha} \phi \simeq \phi + i\alpha \phi$ としても不変 $\delta \mathcal{L} = 0$ であることを示す.

考える Lagrangian は以下の通り:
\begin{equation*}
\mathcal{L} = |\partial_\mu \phi|^2 - m^2 |\phi|^2
= \partial_\mu \phi^* \partial^\mu \phi - m^2 \phi^* \phi. \tag{2-2.g1}
\end{equation*}

無限小のグローバル $U(1)$ 変換を考える:
\begin{align*}
\phi &\to \phi' = e^{i\alpha} \phi \simeq \phi + i\alpha \phi, \tag{2-2.g2}\\
\phi^* &\to \phi'^* = e^{-i\alpha} \phi^* \simeq \phi^* - i\alpha \phi^*, \tag{2-2.g3}
\end{align*}

ここで $\alpha$ は小さな実数パラメータとする.\\
この変換に対して Lagrangian の変化 $\delta \mathcal{L}$ を計算する. 導関数の変化は次の通り:
\begin{align*}
\partial_\mu \phi &\to \partial_\mu \phi' = \partial_\mu (e^{i\alpha} \phi) = e^{i\alpha} \partial_\mu \phi \simeq \partial_\mu \phi + i\alpha \partial_\mu \phi, \tag{2-2.g4}\\
\partial_\mu \phi^* &\to \partial_\mu \phi'^* = \partial_\mu (e^{-i\alpha} \phi^*) = e^{-i\alpha} \partial_\mu \phi^* \simeq \partial_\mu \phi^* - i\alpha \partial_\mu \phi^*. \tag{2-2.g5}
\end{align*}

よって, $|\partial_\mu \phi|^2$ の変化は:
\begin{align*}
|\partial_\mu \phi'|^2 &= (\partial_\mu \phi'^*)(\partial^\mu \phi')  \tag{2-2.g6}\\
&= (\partial_\mu \phi^* - i\alpha \partial_\mu \phi^*)(\partial^\mu \phi + i\alpha \partial^\mu \phi)  \tag{2-2.g7}\\
&= (\partial_\mu \phi^*)(\partial^\mu \phi) + i\alpha (\partial_\mu \phi^*)(\partial^\mu \phi) - i\alpha (\partial_\mu \phi^*)(\partial^\mu \phi) + \mathcal{O}(\alpha^2)  \tag{2-2.g8}\\
&= (\partial_\mu \phi^*)(\partial^\mu \phi) + \mathcal{O}(\alpha^2). \tag{2-2.g9}
\end{align*}

同様に, $|\phi|^2$ の変化は:
\begin{align*}
|\phi'|^2 &= (\phi'^*)(\phi')  \tag{2-2.g10}\\
&= (\phi^* - i\alpha \phi^*)(\phi + i\alpha \phi)  \tag{2-2.g11}\\
&= \phi^* \phi + i\alpha (\phi^* \phi) - i\alpha (\phi^* \phi) + \mathcal{O}(\alpha^2)  \tag{2-2.g12}\\
&= \phi^* \phi + \mathcal{O}(\alpha^2). \tag{2-2.g13}
\end{align*}

以上より, $\mathcal{L}$ の変化は:
\begin{align*}
\mathcal{L}' &= |\partial_\mu \phi'|^2 - m^2 |\phi'|^2  \tag{2-2.g14}\\
&= \left( |\partial_\mu \phi|^2 + \mathcal{O}(\alpha^2) \right) - m^2 \left( |\phi|^2 + \mathcal{O}(\alpha^2) \right)  \tag{2-2.g15}\\
&= \mathcal{L} + \mathcal{O}(\alpha^2). \tag{2-2.g16}
\end{align*}

よって, $\alpha \to 0$ の極限で次のようになる
\begin{equation*}
\delta \mathcal{L} = \mathcal{L}' - \mathcal{L} = 0 \tag{2-2.g17}
\end{equation*}
となり, Lagrangianはこの無限小 $U(1)$ 変換に対して不変であることがわかる.
\vskip\baselineskip
Lagrangian密度の変化 $\delta \mathcal{L}$ は一般に次のように書ける:
\begin{equation*}
\delta \mathcal{L} = \frac{\partial \mathcal{L}}{\partial \phi} \Delta \phi + \frac{\partial \mathcal{L}}{\partial (\partial_\mu \phi)} \partial_\mu (\Delta \phi) \tag{2-2.g18}
\end{equation*}

この式の第二項に積の微分を適用すると:
\begin{equation*}
\frac{\partial \mathcal{L}}{\partial (\partial_\mu \phi)} \partial_\mu (\Delta \phi)
= \partial_\mu \left( \frac{\partial \mathcal{L}}{\partial (\partial_\mu \phi)} \Delta \phi \right) - \left( \partial_\mu \frac{\partial \mathcal{L}}{\partial (\partial_\mu \phi)} \right) \Delta \phi \tag{2-2.g19}
\end{equation*}

したがって,
\begin{equation*}
\delta \mathcal{L}
= \left( \frac{\partial \mathcal{L}}{\partial \phi} - \partial_\mu \frac{\partial \mathcal{L}}{\partial (\partial_\mu \phi)} \right) \Delta \phi
+ \partial_\mu \left( \frac{\partial \mathcal{L}}{\partial (\partial_\mu \phi)} \Delta \phi \right) \tag{2-2.g20}
\end{equation*}

ここで, Euler-Lagrange 方程式
\begin{equation*}
\frac{\partial \mathcal{L}}{\partial \phi} - \partial_\mu \frac{\partial \mathcal{L}}{\partial (\partial_\mu \phi)} = 0 \tag{2-2.g21}
\end{equation*}

を使うと, 第一項が消えるよって,
\begin{equation*}
\delta \mathcal{L} = \partial_\mu \left( \frac{\partial \mathcal{L}}{\partial (\partial_\mu \phi)} \Delta \phi \right) \tag{2-2.g22}
\end{equation*}

ここで, Noether の定理における「表面項」を
\begin{equation*}
\mathcal{J}^\mu = -\frac{\partial \mathcal{L}}{\partial (\partial_\mu \phi)} \Delta \phi \tag{2-2.g23}
\end{equation*}
と定義すると, $\delta \mathcal{L}$ は
\begin{equation*}
\delta \mathcal{L} = -\partial_\mu \mathcal{J}^\mu \tag{2-2.g24}
\end{equation*}
と書ける. したがって, $\delta \mathcal{L} = 0$ ならば, $\partial_\mu \mathcal{J}^\mu = 0$ が成り立つ.\\
さらに, 局所的に $\mathcal{J}^\mu = 0$ としてよい($\mathcal{J}^\mu$ は定数のような無害な全微分項にすぎない).\par
よって, 保存される Noether 電流は次のように求められる:

\begin{align*}
    j^\mu &= \frac{\partial \mathcal{L}}{\partial (\partial_\mu \phi)} \Delta \phi + \frac{\partial \mathcal{L}}{\partial (\partial_\mu \phi^*)} \Delta \phi^* \tag{2-2.g25}\\
    &= \frac{\partial}{\partial(\partial_\mu \phi)}(i\alpha \phi) + \frac{\partial}{\partial(\partial_\mu \phi^*)}(-i\alpha \phi^*) \tag{2-2.g26}\\
    &= (\partial^\mu \phi^*) (i\alpha \phi) + \phi^* (\partial^\mu \phi) (-i\alpha) \tag{2-2.g27}\\
    &= i\alpha \left[ (\partial^\mu \phi^*) \phi - \phi^* (\partial^\mu \phi) \right] \tag{2-2.g28}
\end{align*}

ここで $\alpha$ は変換のパラメータであり, カレントそのものは変換の大きさに依存しない. 保存される量 (電荷など) は対称性の種類には依存するが, どれだけ変換したのか ($\alpha$ の大きさ) には依存しない. よって, $\alpha = 1$ としてよい.

\begin{equation*}
    j^\mu = i \left[ (\partial^\mu \phi^*) \phi - \phi^* (\partial^\mu \phi) \right]. \tag{2-2.g29}
\end{equation*}

\end{proof}

\color{black}

\noindent この電流の発散が, Klein-Gordon 方程式を用いてゼロになることは直接確認できる.

\color{blue}
\vskip\baselineskip
※ この電流の発散が, Klein-Gordon 方程式を用いてゼロになることを確認する.
\begin{align*}
    \partial_\mu j^\mu &= \partial_\mu \left( i \left[ (\partial^\mu \phi^*) \phi - \phi^* (\partial^\mu \phi) \right] \right) \tag{2-2.h1}\\
    &= i \left[ \partial_\mu (\partial^\mu \phi^*) \phi + (\partial^\mu \phi^*) \partial_\mu \phi - \partial_\mu \phi^* (\partial^\mu \phi) - \phi^* \partial_\mu (\partial^\mu \phi) \right] \tag{2-2.h2}\\
    &= i \left[ \partial_\mu (\partial^\mu \phi^*) \phi - \partial_\mu (\partial^\mu \phi) \phi^* \right] \tag{2-2.h3}\\
    &= i \left[  (\partial_\mu \partial^\mu \phi^*) \phi - \phi^* (\partial_\mu \partial^\mu \phi) \right] \tag{2-2.h4}\\
    &= i \left[  (-m^2 \phi^*) \phi - \phi^* (-m^2 \phi) \right] \hspace{0.5cm} (\because \text{Klein-Gordon 方程式}, \partial_\mu \partial^\mu \phi = -m^2 \phi) \tag{2-2.h5}\\
    &= 0. \tag{2-2.h6}
\end{align*}


\color{black}

後にこのラグランジアンに電磁場との相互作用項を追加し, $j^\mu$ を場によって運ばれる電流として解釈する. この場合, $j^0$ は電荷密度に対応し, 空間全体で積分したチャージ $Q$ は場が持つ総電荷となる. \par
Noether の定理は並進や回転などの時空変換にも適応できる. 無限小の平行移動

\begin{equation*}
    x^\mu \to x^\mu + a^\mu
\end{equation*}

を場の構成の変換として表現することができる:

\begin{equation*}
    \phi(x) \to \phi(x+a) = \phi(x) + a^\mu \partial_\mu \phi(x).
\end{equation*}

Lagrangian $\mathcal{L}$ もスカラーであるため, 次のように変換しなければならない:
\begin{equation*}
\mathcal{L} \to \mathcal{L} + a^\mu \partial_\mu \mathcal{L} = \mathcal{L} + a^\nu \partial_\mu (\delta^\mu_\nu \mathcal{L}).
\end{equation*}

この式を \eqref{2.10} と比較すると, ここでゼロでない $\mathcal{J}^\mu$ が現れることがわかる.
これを考慮すると, Noether の定理を適用して4つの独立に保存されるカレントを得られる:

\begin{equation*}\label{2.17}
{T^\mu}_{\nu} \equiv \frac{\partial \mathcal{L}}{\partial (\partial_\mu \phi)} \partial_\nu \phi - \mathcal{L} {\delta^\mu}_{\nu}.
\tag{2.17}
\end{equation*}

これは \textbf{stress-energy tensor} または \textbf{energy-momentum tensor} と呼ばれ, 場 $\phi$ に対応するものである.

\color{blue}

\begin{proof}
\eqref{2.17} の導出.\par
まずは Lagrangian の変化を計算する.
\begin{align*}
    \delta \mathcal{L} &= \underline{\mathcal{L}(\phi + a^\mu \partial_\mu \phi, \partial_\mu \phi + \partial_\nu (a^\nu \partial_\mu \phi))} - \mathcal{L}(\phi, \partial_\mu \phi) \tag{2-2.i1}\\
    &\quad \hspace{0.3cm} \mathcal{L}(\phi, \partial_\mu \phi) + a^\mu \partial_\mu \phi \frac{\partial \mathcal{L}}{\partial \phi} + \partial_\nu (a^\nu \partial_\mu \phi) \frac{\partial \mathcal{L}}{\partial (\partial_\mu \phi)} \tag{2-2.i2}\\
    &= a^\mu \partial_\mu \phi \frac{\partial \mathcal{L}}{\partial \phi} + \underline{\partial_\nu (a^\nu \partial_\mu \phi) \frac{\partial \mathcal{L}}{\partial (\partial_\mu \phi)}} \tag{2-2.i3}\\
    &\quad \hspace{0.3cm} \left[ \underline{\partial_\nu \left( a^\nu \partial_\mu \phi \frac{\partial \mathcal{L}}{\partial (\partial_\mu \phi)} \right)} - a^\nu \partial_\mu \phi \partial_\nu \left( \frac{\partial \mathcal{L}}{\partial (\partial_\mu \phi)} \right) \right] \tag{2-2.i4}\\
    &\quad \hspace{0.3cm} \underline{(\partial_\nu a^\nu)} \partial_\mu \phi \frac{\partial \mathcal{L}}{\partial (\partial_\mu \phi)} + a^\nu \partial_\nu \partial_\mu \phi \frac{\partial \mathcal{L}}{\partial (\partial_\mu \phi)} + a^\nu \partial_\mu \phi \partial_\nu \left( \frac{\partial \mathcal{L}}{\partial (\partial_\mu \phi)} \right) \tag{2-2.i5}\\
    &\quad \hspace{0.3cm} \partial_\nu a^\nu = 0 \hspace{0.5cm} (\because a^\nu \text{は定数パラメータ}) \tag{2-2.i6}\\
    &= a^\mu \partial_\mu \phi \frac{\partial \mathcal{L}}{\partial \phi} + a^\nu \partial_\nu \partial_\mu \phi \left( \frac{\partial \mathcal{L}}{\partial (\partial_\mu \phi)} \right) \tag{2-2.i7}\\
    &= a^{\color{red}\nu} \partial_{\color{red}\nu} \phi \frac{\partial \mathcal{L}}{\partial \phi} + a^\nu \partial_\nu \partial_\mu \phi \left( \frac{\partial \mathcal{L}}{\partial (\partial_\mu \phi)} \right) \hspace{0.3cm} (\text{初項に dummy index を適応}, \mu \to \nu) \tag{2-2.i8}\\
    &= a^\nu \left( \partial_\nu \phi \frac{\partial \mathcal{L}}{\partial \phi} + \underline{\partial_\nu \partial_\mu \phi \left( \frac{\partial \mathcal{L}}{\partial (\partial_\mu \phi)} \right)} \right) \tag{2-2.i9}\\
    &\quad \hspace{1.5cm} \partial_\nu \left( \frac{\partial \mathcal{L}}{\partial (\partial_\mu \phi)}\partial_\mu \phi \right) - \left( \partial_\nu \frac{\partial \mathcal{L}}{\partial (\partial_\mu \phi)} \right) \partial_\mu \phi \tag{2-2.i10}\\
    &= a^\nu \left[ \partial_\nu \left( \frac{\partial \mathcal{L}}{\partial (\partial_\mu \phi)}\partial_\mu \phi \right) - \left( \partial_\nu \frac{\partial \mathcal{L}}{\partial (\partial_\mu \phi)} \right) \partial_\mu \phi + \underline{\partial_\nu \phi \frac{\partial \mathcal{L}}{\partial \phi}} \right] \tag{2-2.i11}\\
    &\quad \hspace{2cm} \frac{\partial \mathcal{L}}{\partial \phi} \partial_\nu \phi = \left( \partial_\mu \frac{\partial \mathcal{L}}{\partial (\partial_\mu \phi)} \right) \partial_\nu \phi \hspace{0.5cm} (\because \text{Euler-Lagrange 方程式}) \tag{2-2.i12}\\
    &\quad \hspace{2cm} \frac{\partial \mathcal{L}}{\partial \phi} \partial_{\color{red}\mu} \phi = \left( \partial_{\color{red}\nu} \frac{\partial \mathcal{L}}{\partial (\partial_\mu \phi)} \right) \partial_{\color{red}\mu} \phi \hspace{0.5cm} (\text{dummy index の入れ替え}, \mu \to \nu, \nu \to \mu.) \tag{2-2.i13}\\
    &= a^\nu \partial_\mu \left( \frac{\partial \mathcal{L}}{\partial (\partial_\mu \phi)} \partial_\nu \phi \right) \tag{2-2.i14}
\end{align*}

一方で, Lagrangian の変化は $\mathcal{L}$ がスカラーであるため, 次のように書ける:
\begin{equation*}
    \delta \mathcal{L} = a^\nu \partial_\nu \mathcal{L} = a^\nu \partial_\mu (\delta^\mu_\nu \mathcal{L}) \tag{2-2.i15}
\end{equation*}

ここで, 両者の変化が等しいことから, 次のようになる:
\begin{align*}
    a^\nu \partial_\mu (\delta^\mu_\nu \mathcal{L}) &= a^\nu \partial_\mu \left( \frac{\partial \mathcal{L}}{\partial (\partial_\mu \phi)} \partial_\nu \phi \right) \tag{2-2.i16}\\
    0 &= a^\nu \partial_\mu \left( \underline{\frac{\partial \mathcal{L}}{\partial (\partial_\mu \phi)} \partial_\nu \phi - \mathcal{L} \delta^\mu_\nu} \right) \tag{2-2.i17}\\
    &\quad \hspace{3cm} \equiv {T^\mu}_{\nu} \tag{2-2.i18}
\end{align*}

\end{proof}

\color{black}

時間並進に対応する保存電荷はハミルトニアンとなる:
\begin{equation*}\label{2.18}
H = \int T^{00} d^3x = \int \mathcal{H} d^3x.
\tag{2.18}
\end{equation*}

\color{blue}

\begin{proof}
\eqref{2.18} での $T^{00}$ の計算.
\begin{align*}
    T^{\mu\nu} &= g^{\nu\rho} {T^\mu}_{\rho} \tag{2-2.j1}\\
    &= g^{\nu\rho} \left( \frac{\partial \mathcal{L}}{\partial (\partial_\mu \phi)} \partial_\rho \phi - \mathcal{L} {\delta^\mu}_\rho \right) \tag{2-2.j2}\\
    &= \frac{\partial \mathcal{L}}{\partial (\partial_\mu \phi)} \partial^\nu \phi - \mathcal{L} \delta^{\mu\nu} \tag{2-2.j3}\\
    T^{00} &= \frac{\partial \mathcal{L}}{\partial (\partial_0 \phi)} \partial^0 \phi - \mathcal{L} \tag{2-2.j4}\\
    &= \frac{\partial \mathcal{L}}{\partial \dot{\phi}} \dot{\phi} - \mathcal{L} \tag{2-2.j5}\\
    &= \frac{\partial}{\partial \dot{\phi}} \left( \frac{1}{2} \dot{\phi}^2 -\frac{1}{2} (\nabla \phi)^2 - \frac{1}{2} m^2 \phi^2 \right) \dot{\phi} - \left( \frac{1}{2} \dot{\phi}^2 -\frac{1}{2} (\nabla \phi)^2 - \frac{1}{2} m^2 \phi^2 \right) \tag{2-2.j6}\\
    &= \frac{1}{2} \dot{\phi}^2 - \frac{1}{2} (\nabla \phi)^2 - \frac{1}{2} m^2 \phi^2 \tag{2-2.j7}\\
    &= \frac{1}{2}\pi^2 + \frac{1}{2} (\nabla \phi)^2 + \frac{1}{2} m^2 \phi^2 \tag{2-2.j8}\\
    &= \mathcal{H} \tag{2-2.j9}
\end{align*}

\end{proof}

\color{black}

Klein-Gordon 場についてこの量を計算すれば, \eqref{2.8} で得られた結果が回収できる.

空間並進に対応する保存電荷は次のように与えられる:
\begin{equation*}\label{2.19}
P^i = \int T^{0i} d^3x = -\int \pi \partial_i \phi \, d^3x.
\tag{2.19}
\end{equation*}

\color{blue}    

\begin{proof}
\eqref{2.19} での $T^{0i}$ の計算($i = 1,2,3$).
\begin{align*}
    T^{\mu\nu} &= \frac{\partial \mathcal{L}}{\partial (\partial_\mu \phi)} \partial^\nu \phi - \mathcal{L} \delta^{\mu\nu} \tag{2-2.k1}\\
    T^{0i} &= -\frac{\partial \mathcal{L}}{\partial (\partial_0 \phi)} \partial^i \phi - \mathcal{L} \delta^{0i} \tag{2-2.k2}\\
    &= -\frac{\partial \mathcal{L}}{\partial \dot{\phi}} \partial^i \phi \tag{2-2.k3}\\
    &= -\frac{\partial}{\partial \dot{\phi}} \left( \frac{1}{2} \dot{\phi}^2 -\frac{1}{2} (\nabla \phi)^2 - \frac{1}{2} m^2 \phi^2 \right) \partial^i \phi \tag{2-2.k4}\\
    &= -\dot{\phi} \partial^i \phi \tag{2-2.k5}\\
    &= -\pi \partial^i \phi \tag{2-2.k6}
\end{align*}
\end{proof}

\color{black}

ここで, $\pi$ は場 $\phi$ に共役な正準運動量密度を表す.\\
この $P^i$ を, 場によって運ばれる(物理的な)運動量と自然に解釈する.\\
(ただし, 正準運動量とは混同しないことに注意).


\end{document}
