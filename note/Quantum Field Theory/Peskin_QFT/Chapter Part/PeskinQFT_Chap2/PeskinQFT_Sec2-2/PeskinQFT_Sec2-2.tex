\documentclass[a4paper,12pt]{article}

\title{Chapter 2. The Klein-Gordon Field\\
2-2. Elements of Classical Field Theory}
\date{各種SNS\\
    X (旧 Twitter): \href{https://x.com/miya_max_study}{@miya\_max\_study}\\
    Instagram : \href{https://www.instagram.com/daily_life_of_miya/}{@daily\_life\_of\_miya}\\
    YouTube : \href{https://www.youtube.com/@miya-max-active}{@miya-max-active}
    }
\author{Max Miyazaki}

\usepackage{amsmath}
\usepackage{amssymb}
\usepackage{ascmac}
\usepackage{amsthm}
\usepackage{amsfonts}
\usepackage{enumitem}
\usepackage{color}
\usepackage[dvipdfmx]{graphicx}
\usepackage{float}
\usepackage{bm}
\usepackage{here}

\usepackage{abstract}
\usepackage{tikz}
\usetikzlibrary{shapes.geometric, arrows.meta, positioning}
\usepackage{indentfirst}
\usepackage[utf8]{inputenc}
\usepackage{fix-cm}
\usepackage{wrapfig}
\pagenumbering{arabic}
\usepackage{url}
\usepackage{xcolor}
\usepackage[most]{tcolorbox}
\usepackage{framed}
\usepackage[dvipdfmx]{hyperref}
\hypersetup{
 setpagesize=false,
 bookmarksnumbered=true,
 colorlinks=true,
 linkcolor=blue
}

% Define braket-like commands
\newcommand{\bra}[1]{\left\langle #1\right|}
\newcommand{\ket}[1]{\left|#1\right\rangle}
\newcommand{\braket}[2]{\left\langle #1\middle|#2\right\rangle}
\newcommand{\brakets}[3]{\left\langle #1\middle| #2 \middle|#3 \right\rangle}

\renewcommand{\arraystretch}{2.1}


\setlength{\textwidth}{16cm}
\setlength{\textheight}{25cm}
\setlength{\oddsidemargin}{0cm}
\setlength{\evensidemargin}{0cm}
\setlength{\topmargin}{-2cm}

\begin{document}
\maketitle

\vspace{1cm}
\begin{abstract}
    このノートはPeskin\&Schroederの``An Introduction to Quantum Field Theory''の第2章の2節をまとめたものである. 要点や個人的な追記, 計算ノート的なまとめを行っているが, それらはすべて原書の内容を出発点としている. 参考程度に使っていただきたいが, このノートは私の勉強ノートであり, そのままの内容をそのまま鵜呑みにすると間違った理解を招く可能性があることをご了承ください. ぜひ原著を手に取り, その内容をご自身で確認していただくことを推奨します. てへぺろ v$({\hat{\cdot}_\partial \hat{\cdot}})$v



\end{abstract}
    
    

\newpage
\section*{2.2 Elements of Classical Field Theory}
\subsection*{Lagrangian Field Theory}

\color{blue}
※ここで登場する $\phi$ は物理空間(時空)の各点 $x$ に定義された『場(field)』である. 具体的には, 各点 $x^\mu = (t, \vec{x})$ に対してある値を取る関数 $\phi(x)$ のことを指す.

このような場は, 次のような観点で理解できる:

\begin{itemize}
    \item $\phi$ は空間や時間の各点において定義される変数で, 例えば温度分布や電場のように『空間全体に広がった物理量』を表現する.
    \item 本節では, $\phi$ を \textbf{スカラー場(scalar field)} として扱う. つまり, 空間や時空の点ごとに \emph{ただ一つの実数値} をとる場を考える.
    \item スカラー場の例としては, 宇宙背景放射の温度分布, あるいは量子場理論におけるヒッグス場などがある.
    \item より一般的にはベクトル場やスピノル場なども存在するが, 基本的な構造を理解するためにはスカラー場が最も扱いやすい.
\end{itemize}

したがって, この節では $\phi(x)$ を \textbf{実数値のスカラー場} (実スカラー場) として議論を進める. この仮定のもとで, Lagrangian 密度 $\mathcal{L}$ は $\phi(x)$ とその導関数 $\partial_\mu \phi(x)$ の関数として定義され, そこから作用 $S$ や運動方程式が導出される.

\color{black}

\begin{itemize}
    \item \textbf{作用} $S$: Lagrangian の時間積分として定義される.
    \begin{equation*}
        S = \int L\,dt = \int \mathcal{L}(\phi, \partial_\mu \phi)\,d^4x \tag{2.1}
    \end{equation*}

    \item \textbf{Lagrangian 密度} $\mathcal{L}$: 場 $\phi(x)$ およびその導関数 $\partial_\mu \phi$ の関数.\\
    \hspace{4cm} \textcolor{blue}{ (以降, この $\mathcal{L}$ を単に Lagrangian と呼ぶ.) }

    \item \textbf{最小作用の原理}:物理系は作用 $S$ が極値(通常は最小)をとる経路に沿って変化.
    
    \begin{align*}
        0 = \delta S 
        &= \int d^4x \left\{ \frac{\partial \mathcal{L}}{\partial \phi} \, \delta \phi 
        + \frac{\partial \mathcal{L}}{\partial (\partial_\mu \phi)} \, \delta(\partial_\mu \phi) \right\} \\
        &= \int d^4x \left\{ \frac{\partial \mathcal{L}}{\partial \phi} \, \delta \phi 
        - \partial_\mu \left( \frac{\partial \mathcal{L}}{\partial (\partial_\mu \phi)} \right) \delta \phi 
        + \partial_\mu \left( \frac{\partial \mathcal{L}}{\partial (\partial_\mu \phi)} \delta \phi \right) \right\}. \tag{2.2}
    \end{align*}
        

    \item \textbf{場に対する Euler-Lagrange 方程式}:
    \begin{equation}
        \partial_\mu \left( \frac{\partial \mathcal{L}}{\partial (\partial_\mu \phi)} \right) - \frac{\partial \mathcal{L}}{\partial \phi} = 0 \tag{2.3}
    \end{equation}
\end{itemize}

\noindent Lagrangian に複数の場が含まれている場合, それぞれの場に対して同様の方程式が存在.

\color{blue}

\begin{proof}
※場の Euler-Lagrange 方程式を最小作用の原理から導出する.

\begin{align*}
    \delta S &= 0 \tag{2-2.a1} \\
    \Rightarrow \delta S &= \int d^4x \left\{ \underline{\mathcal{L}(\phi + \delta \phi, \partial_\mu \phi + \delta \partial_\mu \phi)} - \mathcal{L}(\phi, \partial_\mu \phi) \right\} \tag{2-2.a2} \\
    &\quad \hspace{1.7cm} \mathcal{L}(\phi, \partial_\mu \phi) 
    + \delta \phi \frac{\partial \mathcal{L}}{\partial \phi} 
    + \delta (\partial_\mu \phi) \frac{\partial \mathcal{L}}{\partial (\partial_\mu \phi)} \tag{2-2.a3} \\
    &= \int d^4x \left\{ 
    \delta \phi \frac{\partial \mathcal{L}}{\partial \phi} 
    + \underline{\delta (\partial_\mu \phi) \frac{\partial \mathcal{L}}{\partial (\partial_\mu \phi)}} 
    \right\} \tag{2-2.a4} \\
    &\quad \hspace{1cm} \delta(\partial_\mu \phi)\frac{\partial \mathcal{L}}{\partial(\partial_\mu \phi)}+  \delta \phi \partial_\mu \left( \frac{\partial \mathcal{L}}{\partial (\partial_\mu \phi)} \right) - \delta \phi \partial_\mu \left( \frac{\partial \mathcal{L}}{\partial (\partial_\mu \phi)} \right) \tag{2-2.a5} \\
    &\quad \hspace{1cm} \Leftrightarrow - \delta \phi \partial_\mu \left( \frac{\partial \mathcal{L}}{\partial (\partial_\mu \phi)} \right) + \left\{ \underline{\delta(\partial_\mu \phi)}\, \frac{\partial \mathcal{L}}{\partial(\partial_\mu \phi)} + \delta \phi \partial_\mu \left( \frac{\partial \mathcal{L}}{\partial (\partial_\mu \phi)} \right) \right\} \tag{2-2.a6} \\
    & \quad \hspace{4.7cm} \delta (\partial_\mu \phi) = \partial_\mu (\delta \phi) \tag{2-2.a7} \\
    & \quad \hspace{1cm} \Leftrightarrow - \delta \phi \partial_\mu \left( \frac{\partial \mathcal{L}}{\partial (\partial_\mu \phi)} \right) + \left\{ \partial_\mu(\delta \phi)\, \frac{\partial \mathcal{L}}{\partial(\partial_\mu \phi)} + \delta \phi \partial_\mu \left( \frac{\partial \mathcal{L}}{\partial (\partial_\mu \phi)} \right) \right\} \tag{2-2.a8} \\
    & \quad \hspace{1cm} \Leftrightarrow - \delta \phi \partial_\mu \left( \frac{\partial \mathcal{L}}{\partial (\partial_\mu \phi)} \right) + \partial_\mu \left( \delta \phi \frac{\partial \mathcal{L}}{\partial (\partial_\mu \phi)} \right) \tag{2-2.a9} \\
    &= \int d^4x \left\{ 
    \delta \phi \frac{\partial \mathcal{L}}{\partial \phi} 
    - \delta \phi \partial_\mu \left( \frac{\partial \mathcal{L}}{\partial (\partial_\mu \phi)} \right)
    + \underline{\partial_\mu \left( \delta \phi \frac{\partial \mathcal{L}}{\partial (\partial_\mu \phi)} \right)}
    \right\} \tag{2-2.a10} \\
    &\quad \hspace{6.5cm} \to 0 \hspace{0.5cm}(\because \text{境界条件}\delta \phi_I = \delta \phi_F = 0) \\
    &= \int d^4x  
    \left[ \frac{\partial \mathcal{L}}{\partial \phi} 
    - \partial_\mu \left( \frac{\partial \mathcal{L}}{\partial (\partial_\mu \phi)} \right)
    \right] \delta \phi. \tag{2-2.a11}
\end{align*}

この積分が任意の $\delta \phi$ に対してゼロであるためには括弧内がゼロである必要がある.\\
以上により, 場に対する Euler-Lagrange 方程式が得られる.

\begin{equation*}
    \partial_\mu \left( \frac{\partial \mathcal{L}}{\partial (\partial_\mu \phi)} \right) - \frac{\partial \mathcal{L}}{\partial \phi} = 0 \tag{2-2.a12}
\end{equation*}

\end{proof}

\color{black}

場の運動は作用 $S$ が場の小さな変分に対して不変(すなわち極値を取る)となるように決定される. この条件から Euler-Lagrange 方程式が導かれる.

\subsection*{Hamiltonian Field Theory}

Lagrangian 形式の場の理論は, すべての式が明示的に Lorentz 不変となっているため, 特に相対論的ダイナミクスに適している. しかし, 本書の前半では量子力学への移行を容易にするために Hamiltonian 形式を用いる.
\vskip\baselineskip

\color{blue}
※もう少し Hamiltonian 形式のモチベーションについて述べておく.\\
Hamiltonian 形式では時間を特別な変数として扱う. つまり, $\dot{\phi} = \partial_0 \phi$ を独立変数とするため, 空間と時間の対称的な扱いが失われてしまう. これは時空の対称性 (Lorentz不変性) を明示的に失うことを意味する. だが以下のような利点がある.

\begin{itemize}
    \item 正準量子化 (canonical quantization) がやりやすい.\par
    位置と運動量に対して Poisson 括弧や交換関係をを導入するには時間方向を取り出して『時間発展』の枠組みが必要となる. 
    \item 場の量子論での標準的計算手法に繋がる.\par
    Heisenberg 描像, Feynman 経路積分といった形式に自然に繋がる.
\end{itemize}


\noindent Question:え?でも Lorentz 不変性破ってそんなことして大丈夫なんですか?\\
Answer: Hamiltonian 形式では『一時的に』 Lorentz 不変性を『明示的』に捨てるが, 最終的には理論全体として Lorentz 不変性を回復させることができるので問題ない!

\vskip\baselineskip

理論物理の立場としては,\\
『一時的に非共変になっても良く, 最終的に物理的結果が共変ならOK』\\
Lagrangian 形式も Hamiltonian 形式も
\color{black}

\vskip\baselineskip
離散的な系では, 共役運動量 $p \equiv \partial L / \partial \dot{q}$ を各動的変数 $q$ に対して定義可能. このときハミルトニアンは次のように定義される.

\begin{equation*}
    H \equiv \sum p \dot{q} - L, \hspace{0.5cm} \left(\dot{q} = \frac{\partial q}{\partial t}\right)
\end{equation*}

連続系への一般化は, 空間点 $x$ を離散的に分けられた点であると仮定する.

\noindent 例えば, 以下のように定義する:

\begin{align}
    p(x) &\equiv \frac{\partial L}{\partial \dot{\phi}(x)}
    = \frac{\partial}{\partial \dot{\phi}(x)} \int \mathcal{L}(\phi(y), \dot{\phi}(y))\, d^3y \notag \\
    &\sim \frac{\partial}{\partial \dot{\phi}(x)} \sum_y \mathcal{L}(\phi(y), \dot{\phi}(y))\, d^3y \notag \\
    &= \pi(x) d^3x,
\end{align}

ここで,

\begin{equation}
    \pi(x) \equiv \frac{\partial \mathcal{L}}{\partial \dot{\phi}(x)} \tag{2.4}
\end{equation}

は, $\phi(x)$ に共役な運動量密度(momentum density)と呼ばれる.

このとき, ハミルトニアンは

\begin{equation}
    H = \sum_x p(x) \dot{\phi}(x) - L
\end{equation}

と書ける. これを連続極限に移すと, 次のようになる:

\begin{equation}
    H = \int d^3x \left[ \pi(x) \dot{\phi}(x) - \mathcal{L} \right] 
    \equiv \int d^3x\, \mathcal{H}. \tag{2.5}
\end{equation}

このセクションの後半では, 別の方法を用いてこのハミルトニアン密度 $\mathcal{H}$ の表式を再導出する予定である.

簡単な例として, 単一の場 $\phi(x)$ によって記述される理論を考える. そのときのラグランジアンは以下の通り:

\begin{align}
    \mathcal{L} &= \frac{1}{2} \dot{\phi}^2 - \frac{1}{2} (\nabla \phi)^2 - \frac{1}{2} m^2 \phi^2 \notag \\
    &= \frac{1}{2} (\partial_\mu \phi)^2 - \frac{1}{2} m^2 \phi^2. \tag{2.6}
\end{align}







\end{document}
